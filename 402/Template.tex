% Copyright © 2013 Martin Ueding <dev@martin-ueding.de>

\input{../../header.tex}

\usepackage{placeins}

\newcommand\versuchsnummer{402}

\ihead{physik412 – Versuch \versuchsnummer}
\ifoot{Martin Ueding, Lino Lemmer}

\subject{Praktikumsprotokoll}
\title{Quantelung von Energie}
\subtitle{physik412 – Versuch \versuchsnummer}
\author{
    Martin Ueding \\
    \small{\href{mailto:mu@martin-ueding.de}{mu@martin-ueding.de}}
    \and
    Lino Lemmer \\
    \small{\href{mailto:s6lilemm@uni-bonn.de}{s6lilemm@uni-bonn.de}}
}

%\setcounter{tocdepth}{2}

\begin{document}

\maketitle

\begin{abstract}
    In diesem Versuch wird im ersten Teil mit Hilfe des Photoeffektes das
    Placksche Wirkungsquantum $h$ abgeschätzt. Im zweiten Teil wird die
    Balmer-Serie von Wasserstoff und Deuterium untersucht und durch diese
    ebenfalls das Plancksche Wirkungsquantum bestimmt. Die so erhaltenen Werte
    für $h$ werden verglichen.
\end{abstract}

\tableofcontents

%%%%%%%%%%%%%%%%%%%%%%%%%%%%%%%%%%%%%%%%%%%%%%%%%%%%%%%%%%%%%%%%%%%%%%%%%%%%%%%
%                                   Theorie                                   %
%%%%%%%%%%%%%%%%%%%%%%%%%%%%%%%%%%%%%%%%%%%%%%%%%%%%%%%%%%%%%%%%%%%%%%%%%%%%%%%

\FloatBarrier
\section{Theorie}

%%%%%%%%%%%%%%%%%%%%%%%%%%%%%%%%%%%%%%%%%%%%%%%%%%%%%%%%%%%%%%%%%%%%%%%%%%%%%%%
%                 Bestimmung des Planckschen Wirkungsquantums                 %
%%%%%%%%%%%%%%%%%%%%%%%%%%%%%%%%%%%%%%%%%%%%%%%%%%%%%%%%%%%%%%%%%%%%%%%%%%%%%%%

\FloatBarrier
\section{Bestimmung des Planckschen Wirkungsquantums}

\FloatBarrier
\subsection{Aufbau}

\begin{figure}[htbp]
    \centering
    %\includegraphics[width=\linewidth]{}
    \caption{%
        \cite[Abbildung~P402.1]{physik412-Anleitung}
    }
    \label{fig:P402.1}
\end{figure}

\FloatBarrier
\subsection{Durchführung}

\FloatBarrier
\subsection{Auswertung}

%%%%%%%%%%%%%%%%%%%%%%%%%%%%%%%%%%%%%%%%%%%%%%%%%%%%%%%%%%%%%%%%%%%%%%%%%%%%%%%
%                                Balmer-Serie                                 %
%%%%%%%%%%%%%%%%%%%%%%%%%%%%%%%%%%%%%%%%%%%%%%%%%%%%%%%%%%%%%%%%%%%%%%%%%%%%%%%

\FloatBarrier
\section{Balmer-Serie}

In diesem Versuchsteil untersuchen wir die Balmer-Serie einer
Wasserstoff-Deuterium-Lampe mit einem Reflexionsgitter.

\FloatBarrier
\subsection{Aufbau}

Auf einer optischen Bank mit zwei Armen, deren gegenseitige Lage mit einer
Winkelskala bestimmt werden kann, werden die Komponenten aufgebaut. Dabei wird
auf dem ersten Arm die Lampe, eine Linse mit $f = \SI{50}{\milli\meter}$, ein
einstellbarer Spalt sowie ein Objektiv mit $f = \SI{150}{\milli\meter}$
aufgebaut.

In die Mitte wird das holographische Reflexionsgitter angebracht.

Auf dem zweiten Arm wird von der Mitte aus eine Linse mit $f =
\SI{300}{\milli\meter}$ und ein Okular mit Strichskala angebracht. Im weiteren
Verlauf dieses Teilversuchs wird noch eine CCD-Kamera angebracht.

\FloatBarrier
\subsubsection{Justierung}

Zuerst müssen wir den Aufbau justieren. Dazu benutzen wir die helle
Quecksilberlampe. Diese wird an das Ende des einen Arms gestzt. Mit der ersten
Linse bilden wir die Lampe auf den Spalt ab. Der Spalt ist so angebracht, dass
die schrängen Flanken zur Lichtquelle ausgerichtet sind, und dass der Spalt
senkrecht ist.

Wir setzen das Objektiv zwischen Spalt und Gitter ein und drehen das Gitter so,
dass die Reflexion des Gitters wieder durch das Objektiv fällt und auf den
Spalt trifft. Der Abstand von Objektiv und Spalt ist grob dessen Brennweite $f
= \SI{150}{\milli\meter}$. Mit Hilfe dieser Autokollimationsanordnung stellen
wir die Position des Objektivs so ein, dass das Bild des Spaltes scharf ist.
Dadurch ist der Strahlengang nach dem Objektiv auf unendlich fokussiert.

Nun ist das Bild scharf eingestellt, liegt aber noch ein wenig neben dem Spalt.
Daher drehen wir jetzt das Gitter mit den Rändelschrauben so, dass das Bild
exakt in den Spalt fällt.

Die optische Bank klappen wir nun so zu, dass der Winkel $\omega_\text B$ im
Bereich von \SIrange{130}{170}{\degree} liegt.

Das Okular wird ans Ende des anderen Armes gestellt und so justiert, dass einer
von uns die Skala gut ablesen kann. Zuletzt setzen wir die Linse mit $f =
\SI{300}{\milli\meter}$ ein, damit wir im zweiten Arm die Spektrallinien durch
ein Teleskop beobachten können.

\FloatBarrier
\subsection{Durchführung}

\FloatBarrier
\subsubsection{Bestimmung der Gitterkonstanten}

Wir bestimmen die Gitterkonstante des Gitters durch eine Messung mit einer
Quecksilberdampflampe. Dazu lösen wir die Rändelscchraube der Gittersäule und
drehen so lange, bis die erste Linie sichtbar wird. Um die Linie einfacher
erkennen zu können, weiten wir den Spalt auf und stellen ihn anschließend auf
\SI{0.1}{\milli\meter} zurück.

Mit der Objektivlinse des Beobachtungsteleskops können wir die Linien
scharfstellen, wir justieren so lange, bis diese es sind. Dann lesen wir den
Winkel zwischen den Armen $\omega_\text B$ sowie den Winkel des Gitters
$\omega_\text G$ ab. Die gemessenen Winkel sind in
Tabelle~\ref{tab:messdaten:gitterkonstante}.

Im Anhang der Praktikumsanleitung ist das Spektrum der Quecksilberlampe
angegeben. Diese Tabelle zitieren wir im Anhang~\ref{sec:spektrum}.

\begin{table}[htbp]
    \centering
    \begin{tabular}{SSS}
        {Wellenlänge $\lambda / \si\meter$} & {Winkel $\omega_\text B /
    \si\radian$}  & {Winkel $\omega_\text G / \si\radian$} \\
        \hline
        % TODO Messdaten aus Datei laden.
    \end{tabular}
    \caption{%
        Messdaten für die Bestimmung der Gitterkonstanten mit der
        Quecksilberlampe.
    }
    \label{tab:messdaten:gitterkonstante}
\end{table}

\FloatBarrier
\subsubsection{Untersuchung der Ballmer-Linien mit einem Okular}

\FloatBarrier
\subsubsection{Untersuchung der Ballmer-Linien mit einer CCD-Kamera}

\FloatBarrier
\subsection{Auswertung}

\FloatBarrier
\subsubsection{Bestimmung der Gitterkonstanten}

\FloatBarrier
\subsubsection{Bestimmung der Balmerlinien}

%%%%%%%%%%%%%%%%%%%%%%%%%%%%%%%%%%%%%%%%%%%%%%%%%%%%%%%%%%%%%%%%%%%%%%%%%%%%%%%
%                                 Diskussion                                  %
%%%%%%%%%%%%%%%%%%%%%%%%%%%%%%%%%%%%%%%%%%%%%%%%%%%%%%%%%%%%%%%%%%%%%%%%%%%%%%%

\FloatBarrier
\section{Diskussion}

%%%%%%%%%%%%%%%%%%%%%%%%%%%%%%%%%%%%%%%%%%%%%%%%%%%%%%%%%%%%%%%%%%%%%%%%%%%%%%%
%                               Zusammenfassung                               %
%%%%%%%%%%%%%%%%%%%%%%%%%%%%%%%%%%%%%%%%%%%%%%%%%%%%%%%%%%%%%%%%%%%%%%%%%%%%%%%

\FloatBarrier
\section{Zusammenfassung}

%%%%%%%%%%%%%%%%%%%%%%%%%%%%%%%%%%%%%%%%%%%%%%%%%%%%%%%%%%%%%%%%%%%%%%%%%%%%%%%
%                                   Anhang                                    %
%%%%%%%%%%%%%%%%%%%%%%%%%%%%%%%%%%%%%%%%%%%%%%%%%%%%%%%%%%%%%%%%%%%%%%%%%%%%%%%

\FloatBarrier
\begin{appendix}
    \section{Spektrum der Quecksilberlampe}
    \label{sec:spektrum}

    \begin{table}[htbp]
        \centering
        \begin{tabular}{lSS}
            Farbe & {$\lambda_\text{Hg} / \si{\nano\meter}$} & {rel. Int.} \\
            \hline
            violett & 404.656 & 1800 \\
                    & 407.783 & 150 \\
                    & 410.805 & 40 \\
                    & 433.922 & 250 \\
                    & 434.749 & 500 \\
            blau & 435.833 & 4000 \\
            türkis & 491.607 & 80 \\
            grün & 546.074 & 1100 \\
            gelb & 576.960 & 240 \\
                 & 579.066 & 280 \\
            rot & 623.440 & 30 \\
                & 672.643 & 160 \\
                & 690.752 & 250
        \end{tabular}
        \caption{%
            Spektrum der Quecksilberlampe.
            \cite[P402.6.1]{physik412-Anleitung}
        }
        \label{tab:messdaten:gitterkonstante}
    \end{table}



    \section{\LaTeX-Quelltext}

    Der \LaTeX-Quelltext zu allen Protokollen in diesem Praktikum kann auf
    \ref{it:mu} eingesehen werden. Die Quellen für alle Protokolle in diesem
    Praktikum können auf \ref{it:github/alles} eingesehen werden. Die
    \LaTeX-Datei wird aus \ref{it:github/template} generiert.

    \begin{enumerate}
        \item
            \label{it:mu}
            \url{http://martin-ueding.de/de/university/physik412/}
        \item
            \label{it:github/alles}
            \url{https://github.com/martin-ueding/physik412-Protokolle/}
        \item
            \label{it:github/template}
            \url{https://github.com/martin-ueding/physik412-Protokolle/blob/master/\versuchsnummer/Template.tex}
    \end{enumerate}
\end{appendix}


%%%%%%%%%%%%%%%%%%%%%%%%%%%%%%%%%%%%%%%%%%%%%%%%%%%%%%%%%%%%%%%%%%%%%%%%%%%%%%%
%                                  Literatur                                  %
%%%%%%%%%%%%%%%%%%%%%%%%%%%%%%%%%%%%%%%%%%%%%%%%%%%%%%%%%%%%%%%%%%%%%%%%%%%%%%%

\FloatBarrier
\IfFileExists{\bibliographyfile}{
    \bibliography{\bibliographyfile}
}{}

\end{document}

% vim: et spell spelllang=de tw=79
