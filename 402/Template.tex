% Copyright © 2013 Martin Ueding <dev@martin-ueding.de>

\input{../../header.tex}

\usepackage{placeins}

\newcommand\versuchsnummer{402}

\ihead{physik412 – Versuch \versuchsnummer}
\ifoot{Martin Ueding, Lino Lemmer}

\subject{Praktikumsprotokoll}
\title{Quantelung von Energie}
\subtitle{physik412 – Versuch \versuchsnummer}
\author{
    Martin Ueding \\
    \small{\href{mailto:mu@martin-ueding.de}{mu@martin-ueding.de}}
    \and
    Lino Lemmer \\
    \small{\href{mailto:s6lilemm@uni-bonn.de}{s6lilemm@uni-bonn.de}}
}

%\setcounter{tocdepth}{2}

\begin{document}

\maketitle

\newpage
\tableofcontents
\newpage

%%%%%%%%%%%%%%%%%%%%%%%%%%%%%%%%%%%%%%%%%%%%%%%%%%%%%%%%%%%%%%%%%%%%%%%%%%%%%%%
%                                 Einleitung                                  %
%%%%%%%%%%%%%%%%%%%%%%%%%%%%%%%%%%%%%%%%%%%%%%%%%%%%%%%%%%%%%%%%%%%%%%%%%%%%%%%

\section{Einleitung}

In diesem Versuch wird im ersten Teil mit Hilfe des Photoeffektes das Placksche
Wirkungsquantum abgeschätzt. Im zweiten Teil wird die Balmer-Serie von
Wasserstoff und Deuterium untersucht und durch diese ebenfalls das Plancksche
Wirkungsquantum bestimmt. Die so erhaltenen Werte werden verglichen.

\section{Theorie}

\subsection{Photoeffekt}

Hinreichend kurzwelliges, also energiereiches Licht ist in der Lage, Elektronen
aus der Oberfläche eines Metalls auszulösen, indem einzelne Lichtquanten ihre
Energie $E = h\nu$ vollständig an ein Elektron im Metall abgeben. Die Energie
muss dabei die Austrittsarbeit $W_\text{A}$ Übersteigen, die übrige Energie
behält das Elektron in Form von kinetischer Energie. Die Menge der ausgelösten
Elektronen hängt dabei von der Intensität des Lichtes ab.

\subsection{Photozelle}

Der Aufbau einer Fotozelle ist in Abbildung~\ref{fig:Photozelle} zu sehen.

\subsection{Aufbau der Atomhülle}

\subsection{Spektroskopie}

\section{Versuchsdurchführung}

\section{Messung}

\section{Auswertung}

\section{Ergebnisse/Diskussion}

\begin{appendix}
    \section{\LaTeX-Quelltext}

    Der \LaTeX-Quelltext zu allen Protokollen in diesem Praktikum kann auf
    \ref{it:mu} eingesehen werden. Die Quellen für dieses Protokoll können auf
    \ref{it:github/alles} eingesehen werden. Die \LaTeX-Datei wird aus
    \ref{it:github/template} generiert.

    \begin{enumerate}
        \item
            \label{it:mu}
            \url{http://martin-ueding.de/de/university/physik412/}
        \item
            \label{it:github/alles}
            \url{https://github.com/martin-ueding/physik412-Protokolle/}
        \item
            \label{it:github/template}
            \url{https://github.com/martin-ueding/physik412-Protokolle/blob/master/\versuchsnummer/Template.tex}
    \end{enumerate}
\end{appendix}


%%%%%%%%%%%%%%%%%%%%%%%%%%%%%%%%%%%%%%%%%%%%%%%%%%%%%%%%%%%%%%%%%%%%%%%%%%%%%%%
%                                  Literatur                                  %
%%%%%%%%%%%%%%%%%%%%%%%%%%%%%%%%%%%%%%%%%%%%%%%%%%%%%%%%%%%%%%%%%%%%%%%%%%%%%%%

\FloatBarrier
\IfFileExists{\bibliographyfile}{
    \bibliography{\bibliographyfile}
}{}

\end{document}

% vim: et spell spelllang=de tw=79
