% Copyright © 2013 Martin Ueding <dev@martin-ueding.de>

\input{../../header.tex}

\usepackage{placeins}

\newcommand\versuchsnummer{402}

\ihead{physik412 – Versuch \versuchsnummer}
\ifoot{Martin Ueding, Lino Lemmer}

\subject{Praktikumsprotokoll}
\title{Quantelung von Energie}
\subtitle{physik412 – Versuch \versuchsnummer}
\author{
    Martin Ueding \\
    \small{\href{mailto:mu@martin-ueding.de}{mu@martin-ueding.de}}
    \and
    Lino Lemmer \\
    \small{\href{mailto:s6lilemm@uni-bonn.de}{s6lilemm@uni-bonn.de}}
}

%\setcounter{tocdepth}{2}

\begin{document}

\maketitle

\begin{abstract}
    In diesem Versuch wird im ersten Teil mit Hilfe des Photoeffektes das
    Placksche Wirkungsquantum $h$ abgeschätzt. Im zweiten Teil wird die
    Balmer-Serie von Wasserstoff und Deuterium untersucht und durch diese
    ebenfalls das Plancksche Wirkungsquantum bestimmt. Die so erhaltenen Werte
    für $h$ werden verglichen.
\end{abstract}

\tableofcontents

%%%%%%%%%%%%%%%%%%%%%%%%%%%%%%%%%%%%%%%%%%%%%%%%%%%%%%%%%%%%%%%%%%%%%%%%%%%%%%%
%                                   Theorie                                   %
%%%%%%%%%%%%%%%%%%%%%%%%%%%%%%%%%%%%%%%%%%%%%%%%%%%%%%%%%%%%%%%%%%%%%%%%%%%%%%%

\FloatBarrier
\section{Theorie}

%%%%%%%%%%%%%%%%%%%%%%%%%%%%%%%%%%%%%%%%%%%%%%%%%%%%%%%%%%%%%%%%%%%%%%%%%%%%%%%
%                 Bestimmung des Planckschen Wirkungsquantums                 %
%%%%%%%%%%%%%%%%%%%%%%%%%%%%%%%%%%%%%%%%%%%%%%%%%%%%%%%%%%%%%%%%%%%%%%%%%%%%%%%

\FloatBarrier
\section{Bestimmung des Planckschen Wirkungsquantums}

\FloatBarrier
\subsection{Aufbau}

\begin{figure}[htbp]
    \centering
    %\includegraphics[width=\linewidth]{}
    \caption{%
        \cite[Abbildung~P402.1]{physik412-Anleitung}
    }
    \label{fig:P402.1}
\end{figure}

\FloatBarrier
\subsection{Durchführung}

\FloatBarrier
\subsection{Auswertung}

%%%%%%%%%%%%%%%%%%%%%%%%%%%%%%%%%%%%%%%%%%%%%%%%%%%%%%%%%%%%%%%%%%%%%%%%%%%%%%%
%                                Balmer-Serie                                 %
%%%%%%%%%%%%%%%%%%%%%%%%%%%%%%%%%%%%%%%%%%%%%%%%%%%%%%%%%%%%%%%%%%%%%%%%%%%%%%%

\FloatBarrier
\section{Balmer-Serie}

\FloatBarrier
\subsection{Aufbau}

\FloatBarrier
\subsubsection{Justierung}

\FloatBarrier
\subsection{Durchführung}

\FloatBarrier
\subsubsection{Bestimmung der Gitterkonstanten}

Wir bestimmen die Gitterkonstante des Gitters durch eine Messung mit einer
Quecksilberdampflampe. Dazu lösen wir die Rändelscchraube der Gittersäule und
drehen so lange, bis die erste Linie sichtbar wird. Um die Linie einfacher
erkennen zu können, weiten wir den Spalt auf und stellen ihn anschließend auf
\SI{0.1}{\milli\meter} zurück.

Im Anhang der Praktikumsanleitung ist das Spektrum der Quecksilberlampe
angegeben. Diese Tabelle zitieren wir im Anhang~\ref{sec:spektrum}.

\begin{table}[htbp]
    \centering
    \begin{tabular}{SS}
        {Wellenlänge $\lambda / \si\meter$} & {Winkel $\omega_\text B /
    \si\radian$} \\
        \hline
    \end{tabular}
    \caption{%
    }
    \label{tab:messdaten:gitterkonstante}
\end{table}

\FloatBarrier
\subsubsection{Untersuchung der Ballmer-Linien mit einem Okular}

\FloatBarrier
\subsubsection{Untersuchung der Ballmer-Linien mit einer CCD-Kamera}

\FloatBarrier
\subsection{Auswertung}

\FloatBarrier
\subsubsection{Bestimmung der Gitterkonstanten}

\FloatBarrier
\subsubsection{Bestimmung der Balmerlinien}

%%%%%%%%%%%%%%%%%%%%%%%%%%%%%%%%%%%%%%%%%%%%%%%%%%%%%%%%%%%%%%%%%%%%%%%%%%%%%%%
%                                 Diskussion                                  %
%%%%%%%%%%%%%%%%%%%%%%%%%%%%%%%%%%%%%%%%%%%%%%%%%%%%%%%%%%%%%%%%%%%%%%%%%%%%%%%

\FloatBarrier
\section{Diskussion}

%%%%%%%%%%%%%%%%%%%%%%%%%%%%%%%%%%%%%%%%%%%%%%%%%%%%%%%%%%%%%%%%%%%%%%%%%%%%%%%
%                               Zusammenfassung                               %
%%%%%%%%%%%%%%%%%%%%%%%%%%%%%%%%%%%%%%%%%%%%%%%%%%%%%%%%%%%%%%%%%%%%%%%%%%%%%%%

\FloatBarrier
\section{Zusammenfassung}

%%%%%%%%%%%%%%%%%%%%%%%%%%%%%%%%%%%%%%%%%%%%%%%%%%%%%%%%%%%%%%%%%%%%%%%%%%%%%%%
%                                   Anhang                                    %
%%%%%%%%%%%%%%%%%%%%%%%%%%%%%%%%%%%%%%%%%%%%%%%%%%%%%%%%%%%%%%%%%%%%%%%%%%%%%%%

\FloatBarrier
\begin{appendix}
    \section{Spektrum der Quecksilberlampe}
    \label{sec:spektrum}

    \begin{table}[htbp]
        \centering
        \begin{tabular}{lSS}
            Farbe & {$\lambda_\text{Hg} / \si{\nano\meter}$} & {rel. Int.} \\
            \hline
            violett & 404.656 & 1800 \\
                    & 407.783 & 150 \\
                    & 410.805 & 40 \\
                    & 433.922 & 250 \\
                    & 434.749 & 500 \\
            blau & 435.833 & 4000 \\
            türkis & 491.607 & 80 \\
            grün & 546.074 & 1100 \\
            gelb & 576.960 & 240 \\
                 & 579.066 & 280 \\
            rot & 623.440 & 30 \\
                & 672.643 & 160 \\
                & 690.752 & 250
        \end{tabular}
        \caption{%
            Spektrum der Quecksilberlampe.
            \cite[P402.6.1]{physik412-Anleitung}
        }
        \label{tab:messdaten:gitterkonstante}
    \end{table}



    \section{\LaTeX-Quelltext}

    Der \LaTeX-Quelltext zu allen Protokollen in diesem Praktikum kann auf
    \ref{it:mu} eingesehen werden. Die Quellen für dieses Protokoll können auf
    \ref{it:github/alles} eingesehen werden. Die \LaTeX-Datei wird aus
    \ref{it:github/template} generiert.

    \begin{enumerate}
        \item
            \label{it:mu}
            \url{http://martin-ueding.de/de/university/physik412/}
        \item
            \label{it:github/alles}
            \url{https://github.com/martin-ueding/physik412-Protokolle/}
        \item
            \label{it:github/template}
            \url{https://github.com/martin-ueding/physik412-Protokolle/blob/master/\versuchsnummer/Template.tex}
    \end{enumerate}
\end{appendix}


%%%%%%%%%%%%%%%%%%%%%%%%%%%%%%%%%%%%%%%%%%%%%%%%%%%%%%%%%%%%%%%%%%%%%%%%%%%%%%%
%                                  Literatur                                  %
%%%%%%%%%%%%%%%%%%%%%%%%%%%%%%%%%%%%%%%%%%%%%%%%%%%%%%%%%%%%%%%%%%%%%%%%%%%%%%%

\FloatBarrier
\IfFileExists{\bibliographyfile}{
    \bibliography{\bibliographyfile}
}{}

\end{document}

% vim: et spell spelllang=de tw=79
