% Copyright © 2013 Martin Ueding <dev@martin-ueding.de>

\input{../../header.tex}

\usepackage{booktabs}
\usepackage[section]{placeins}

\newcommand\versuchsnummer{428}

\ihead{physik412 – Versuch \versuchsnummer}
\ifoot{Martin Ueding, Lino Lemmer}

\subject{Praktikumsprotokoll}
\title{Röntgenstrahlung und Materialanalyse}
\subtitle{physik412 – Versuch \versuchsnummer}
\author{
    Martin Ueding \\
    \small{\href{mailto:mu@martin-ueding.de}{mu@martin-ueding.de}}
    \and
    Lino Lemmer \\
    \small{\href{mailto:s6lilemm@uni-bonn.de}{s6lilemm@uni-bonn.de}}
}

\setcounter{secnumdepth}{4}
\setcounter{tocdepth}{4}

\begin{document}

\maketitle

\begin{abstract}
    \fehlt
\end{abstract}

\tableofcontents

\chapter{Theorie}

\section{Röntgenstrahlung}

Als Röntgenstrahlung bezeichnet man den Bereich im elektromagnetischen
Spektrum mit Wellenlängen zwischen etwa \SI{10}{\nano\meter} und
\SI{1}{\pico\meter}, also mit einer Photonenenergie zwischen
\SI{100}{\electronvolt} und einigen \si{\mega\electronvolt}. Damit liegt sie
zwischen der langwelligeren UV-Strahlung und der $\gamma$-Strahlung. Die
Abgrenzung zu letzterer ist fließend, die Bereiche überlappen sich zum
Teil. Sie unterscheiden sich in erster Linie durch ihren Entstehungsort:
Während Röntgenstrahlung in der Atomhülle entsteht, also hauptsächlich
durch Coulomb-Wechselwirkung, kommt $\gamma$-Strahlung aus dem Atomkern, wo
schwache und starke Wechselwirkung ebenso eine Rolle spielen.

\section{Entstehung der Röntgenstrahlung}


\parencite[Abschnitt~17.3.1]{meschede-gerthsen_24}

\subsection{Bremsstrahlung}

Die erste ist die Bremstrahlung. Stoßen schnelle Elektronen auf Materie,
werden sie Abgebremst. Da es geladene Teilchen sind, wird dabei
Bremsstrahlung frei. Bei genügend schnellen Elektronen ist die Wellenlänge 
im Röntgenbereich. Das so entstehende Spektrum ist kontinuierlich. Im
Experiment kann die minimale Wellenlänge durch Variation der
Beschleunigungsspannung verändert werden.

\parencite[Abschnitt~17.3.4]{meschede-gerthsen_24}

\subsection{$K_\alpha$ und weitere Linien}

Prallen energiereiche Elektronen auf Materie, können durch inelastische
Stöße an Atomkerne gebundene Elektronen aus ihren Bahnen geschlagen
werden. Die dabei entstehenden Löcher werden durch energetisch höher
liegende Elektronen gefüllt, welche beim Übergang Energie abgeben. Das so
entstehende Spektrum ist diskret und charakteristisch für jede Atomart. Die
stattfindenden Übergänge werden nach der neu gefüllten Schale und dem
Abstand der Energieniveaus benannt. Wird das niedrigste Energieniveau, also
die $K$-Schale, durch ein Elektron aus der $L$-Schale gefüllt, so nennt man
die entstehende Linie $K_\alpha$, wird sie aus der $M$-Schale gefüllt, ist
das Ergebnis die $K_\beta$-Linie.

\parencite[Abschnitt~17.3.5]{meschede-gerthsen_24}

\section{Nachweis von Röntgenstrahlung}

\subsection{Geiger-Müller Zählrohr}

\parencite[Abschnitt~9.3.2]{meschede-gerthsen_24}

\parencite[Abschnitt~19.3.2~d)]{meschede-gerthsen_24}

\FloatBarrier
\subsubsection{Aufbau}

\begin{figure}[htpb]
    \centering
    \includegraphics[width=\textwidth]{../Abbildungen/Geiger-Mueller.pdf}
    \caption{%
        Aufbau eines Geiger-Müller-Zählers
    }
    \label{fig:geiger}
\end{figure}

Der Aufbau eines Geiger-Müller-Zählrohres ist in Abbildung~\ref{fig:geiger} 
zu sehen. Sowohl die Spannung, als auch der Widerstand sind groß. Die 
Kammer mit dem etwa \SI{0.1}{\milli\meter} dicken Anodendraht ist mit einem 
Gasgemisch gefült, häufig werden Edelgase oder einfache Luft mit 
Ethanoldampf vermischt.

\FloatBarrier
\subsubsection{Funktionsweise}

Gelangt Strahlung in das Rohr werden vereinzelt Gasmoleküle ionisiert. Was
nun passiert kommt auf die angelegte Spannung an. Bei geringer Spannung
rekombinieren die Elektronen auf dem Weg zur Anode, was dafür sorgt, dass das
Signal keine brauchbaren Informationen liefert.

Bei einigen \SI{100}{\volt} lösen Elektronen nah der Anode durch Ionisation
weitere Ionen aus, es entstehen Elektronenlawinen mit bis zu \num{1e6}
Elektronen. Das entstehende Signal ist leicht messbar und proportional zur
durch die Strahlung im Zählrohr abgegebenen Energie. Daher wird das
Zählrohr in diesem Betrieb auch Proportionalzählrohr genannt.

Ab einer noch höheren Spannung erzeugt jede einfallende ionisierende
Strahlung Elektronenlawinen, die sich durch das ganze Rohr erstrecken. Das
entstehende Signal ist so groß, dass es ohne weitere Verstärkung zum
Beispiel in einem Lautsprecher ein Knacken erzeugen kann. In diesem Betrieb ist
das Zählrohr maximal empfindlich.

Ist die Spannung noch größer, braucht es keine einfallende Strahlung mehr um
das Gas zu Ionisieren, dies geschieht durch die Spannung allein. Das
Zählrohr ist in diesem Betrieb nicht mehr brauchbar.

\subsubsection{Totzeit}

\subsection{Röntgenenergiedetektor}

\subsubsection{Aufbau}

\subsubsection{Funktionsweise}

\subsubsection{PIN-Photodiode}

\parencite[Abschnitt~„Photodiode“]{wikipedia/pin-Diode}

\subsubsection{Vielkanalanalysator}

Ein Vielkanalanalysator registriert die Spannungspulse, teilt diese in viele
Spannungsintervalle auf und zählt die Häufigkeit.
\parencite{Phywe/Vierkanalanalysator} So entsteht ein Histogram der Energien.
\parencite{wikipedia/Vielkanalanalysator}

\section{Röntgenfluoreszenz}

\subsection{Bestimmung der Massenanteile}

\parencite[„Massenanteilsbestimmung“]{physik412-Anleitung}

\section{Bragg-Beugung}

\parencite[(18.4)]{meschede-gerthsen_24}
\[
    \Deltaup \vec k = \vec g
\]

\section{Laue-Verfahren}

\parencite[P428.5.3, „Auswertung“]{physik412-Anleitung}

\subsection{Laue-Bedingung}

\subsection{Miller'sche Indizes}

\subsection{Elementarzelle}

\subsection{Glanzwinkel}

\subsection{Netzebenenabstand}

\chapter{Bragg-Reflexion}

\section{Aufbau}

\parencite{wikipedia/Goniometer}
\parencite{wikipedia/hygroskopie}
\parencite{leybold/554831}
\parencite{leybold/554800}

Wir stellen den Taster auf \textsc{coupled}, damit der Winkel des Zählrohrs immer
der doppelte Winkel der Probe ist. \parencite{leybold/554800} Auf diese Weise
erhalten wir eine optimale Bragg-Beugung und maximale Intensität.

\section{Bestimmung des Anodenmaterials}

\begin{figure}[htbp]
    \centering
    \includegraphics[width=\linewidth]{Roehre_2.pdf}
    \caption{%
        Spektrum der unbekannten Röhre mit Nummer 2.
    }
    \label{fig:}
\end{figure}

\subsection{Auswertung}

\section{Feinstrukturmessung}

\begin{figure}[htbp]
    \centering
    \includegraphics[width=\linewidth]{Langzeit.pdf}
    \caption{%
        Spektrum der Molybdän-Röhre. Es ist die Aufspaltung der
        $K_\alpha$-Linie in vierter Beugungsordnung zu sehen.
    }
    \label{fig:}
\end{figure}

\subsection{Auswertung}

\chapter{Zerstörungsfreie Aufnahme chemischer Zusammensetzungen}

\section{Aufbau}

\subsection{Tausch der Röntgenröhre}

\subsection{Einbau Energiedetektor}

\section{Kalibrierung mit FeZn}

\subsection{Durchführung}

\subsection{Auswertung}

\begin{figure}[htbp]
    \centering
    \includegraphics[width=\linewidth]{FeZn.pdf}
    \caption{%
        Spektrum von FeZn.
    }
    \label{fig:FeZn}
\end{figure}

\begin{figure}[htbp]
    \centering
    \includegraphics[width=\linewidth]{Energieeichung.pdf}
    \caption{%
        Anpassung der Energie an die Kanalnummer.
    }
    \label{fig:}
\end{figure}

\section{Aufnahme der Legierung}

\subsection{Durchführung}

\subsection{Auswertung}

\begin{figure}[htbp]
    \centering
    %\includegraphics[width=\linewidth]{Probe_1.pdf}
    \caption{%
        Spektrum der unbekannten Probe 1.
    }
    \label{fig:}
\end{figure}

\section{Aufnahme der reinen Metalle}

\subsection{Durchführung}

\subsection{Auswertung}

\section{Bestimmung der Legierung und Massenanteile}

\begin{figure}[htbp]
    \centering
    \includegraphics[width=\linewidth]{Bestimmung.pdf}
    \caption{%
        Spektra der unbekannten Probe sowie der reinen Metalle.
    }
    \label{fig:Bestimmung}
\end{figure}

\begin{figure}[htbp]
    \centering
    \includegraphics[width=\linewidth]{Bestimmung_Zoom.pdf}
    \caption{%
        Ausschnittsvergrößerung von Abbildung~\ref{fig:Bestimmung}
    }
    \label{fig:Bestimmung_Zoom}
\end{figure}

\chapter{Laue-Aufnahme}

\section{Aufbau}

\section{Durchführung}

\subsection{Entwickeln des Films}

\begin{figure}[htbp]
    \centering
    \includegraphics[width=\linewidth]{../Messdaten/laue.png}
    \caption{%
        Scan des Fotos, das wir durch Beugung an einem NaCl-Kristall erzeugt
        haben.
    }
    \label{fig:Laue}
\end{figure}

\section{Auswertung}

\begin{figure}[htbp]
    \centering
    \includegraphics[width=\linewidth]{laue-helligkeit.pdf}
    \caption{%
        Helligkeit im blauen Kanal.
    }
    \label{fig:}
\end{figure}

\begin{figure}[htbp]
    \centering
    \includegraphics[width=\linewidth]{laue-helligkeit-gauss.pdf}
    \caption{%
        Helligkeit nach Gauss'scher Unschärfe mit << gauss_radius >> Pixeln
        Radius.
    }
    \label{fig:}
\end{figure}

\subsection{Ablesen der Reflexpunkte}

\begin{figure}[htbp]
    \centering
    \includegraphics[width=\linewidth]{laue-gradient_x_y.pdf}
    \caption{%
        Amplitude des Gradients
    }
    \label{fig:}
\end{figure}

\begin{figure}[htbp]
    \centering
    \includegraphics[width=\linewidth]{laue-laplace}
    \caption{%
        Laplace
    }
    \label{fig:}
\end{figure}

\begin{figure}[htbp]
    \centering
    \includegraphics[width=\linewidth]{laue-peaks.pdf}
    \caption{%
        Gefundene Maxima
    }
    \label{fig:}
\end{figure}

\subsection{Zuordnung der Gitterebenen}

\chapter{Zusammenfassung}

\chapter{Diskussion}

%%%%%%%%%%%%%%%%%%%%%%%%%%%%%%%%%%%%%%%%%%%%%%%%%%%%%%%%%%%%%%%%%%%%%%%%%%%%%%%
%                                   Anhang                                    %
%%%%%%%%%%%%%%%%%%%%%%%%%%%%%%%%%%%%%%%%%%%%%%%%%%%%%%%%%%%%%%%%%%%%%%%%%%%%%%%

\FloatBarrier
\begin{appendix}
    \FloatBarrier
    \chapter{\LaTeX-Quelltext}

    Der \LaTeX-Quelltext zu allen Protokollen in diesem Praktikum kann auf
    \ref{it:mu} eingesehen werden. Die Quellen für alle Protokolle in diesem
    Praktikum können auf \ref{it:github/alles} eingesehen werden. Die
    \LaTeX-Datei wird aus \ref{it:github/template} generiert.

    \begin{enumerate}
        \item
            \label{it:mu}
            \url{http://martin-ueding.de/de/university/physik412/}
        \item
            \label{it:github/alles}
            \url{https://github.com/martin-ueding/physik412-Protokolle/}
        \item
            \label{it:github/template}
            \url{https://github.com/martin-ueding/physik412-Protokolle/blob/master/\versuchsnummer/Template.tex}
    \end{enumerate}
\end{appendix}

%%%%%%%%%%%%%%%%%%%%%%%%%%%%%%%%%%%%%%%%%%%%%%%%%%%%%%%%%%%%%%%%%%%%%%%%%%%%%%%
%                                  Literatur                                  %
%%%%%%%%%%%%%%%%%%%%%%%%%%%%%%%%%%%%%%%%%%%%%%%%%%%%%%%%%%%%%%%%%%%%%%%%%%%%%%%

\FloatBarrier
\printbibliography

\end{document}

% vim: et spell spelllang=de tw=79
