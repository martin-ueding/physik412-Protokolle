% Copyright © 2013 Martin Ueding <dev@martin-ueding.de>

\input{../../header.tex}

\usepackage{booktabs}
\usepackage[section]{placeins}

\newcommand\versuchsnummer{428}

\ihead{physik412 – Versuch \versuchsnummer}
\ifoot{Martin Ueding, Lino Lemmer}

\subject{Praktikumsprotokoll}
\title{Röntgenstrahlung und Materialanalyse}
\subtitle{physik412 – Versuch \versuchsnummer}
\author{
    Martin Ueding \\
    \small{\href{mailto:mu@martin-ueding.de}{mu@martin-ueding.de}}
    \and
    Lino Lemmer \\
    \small{\href{mailto:s6lilemm@uni-bonn.de}{s6lilemm@uni-bonn.de}}
}

\setcounter{secnumdepth}{4}
\setcounter{tocdepth}{4}

\begin{document}

\maketitle

\begin{abstract}
    \fehlt
\end{abstract}

\tableofcontents

\chapter{Theorie}

\section{Röntgenstrahlung}

Als Röntgenstrahlung bezeichnet man den Bereich im elektromagnetischen
Spektrum mit Wellenlängen zwischen etwa \SI{10}{\nano\meter} und
\SI{1}{\piko\meter}, also mit einer Photonenenergie zwischen
\SI{100}{\electronvolt} und einigen \si{\mega\electronvolt}. Damit liegt sie
zwischen der langwelligeren UV-Strahlung und der $\gamma$-Strahlung. Die
Abgrenzung zu letzterer ist fließend, die Bereiche überlappen sich zum
Teil. Sie unterscheiden sich in erster Linie durch ihren Entstehungsort:
Während Röntgenstrahlung in der Atomhülle entsteht, also hauptsächlich
durch Coulomb-Wechselwirkung, kommt $\gamma$-Strahlung aus dem Atomkern, wo
schwache und starke Wechselwirkung ebenso eine Rolle spielen.

\section{Entstehung der Röntgenstrahlung}

\parencite[Abschnitt~17.3.1]{meschede-gerthsen_24}

\subsection{Bremsstrahlung}

\parencite[Abschnitt~17.3.4]{meschede-gerthsen_24}

\subsection{$K_\alpha$ und weitere Linien}

\parencite[Abschnitt~17.3.5]{meschede-gerthsen_24}

\section{Nachweis von Röntgenstrahlung}

\subsection{Geiger-Müller Zählrohr}

\parencite[Abschnitt~9.3.2]{meschede-gerthsen_24}

\parencite[Abschnitt~19.3.2~d)]{meschede-gerthsen_24}

\subsubsection{Aufbau}

\subsubsection{Funktionsweise}

\subsubsection{Totzeit}

\subsection{Röntgenenergiedetektor}

\subsubsection{Aufbau}

\subsubsection{Funktionsweise}

\subsubsection{PIN-Photodiode}

\parencite[Abschnitt~„Photodiode“]{wikipedia/pin-Diode}

\subsubsection{Vielkanalanalysator}

Ein Vielkanalanalysator registriert die Spannungspulse, teilt diese in viele
Spannungsintervalle auf und zählt die Häufigkeit.
\parencite{Phywe/Vierkanalanalysator} So entsteht ein Histogram der Energien.
\parencite{wikipedia/Vielkanalanalysator}

\section{Röntgenfluoreszenz}

\subsection{Bestimmung der Massenanteile}

\parencite[„Massenanteilsbestimmung“]{physik412-Anleitung}

\section{Bragg-Beugung}

\parencite[(18.4)]{meschede-gerthsen_24}
\[
    \Deltaup \vec k = \vec g
\]

\section{Laue-Verfahren}

\parencite[P428.5.3, „Auswertung“]{physik412-Anleitung}

\subsection{Laue-Bedingung}

\subsection{Miller'sche Indizes}

\subsection{Elementarzelle}

\subsection{Glanzwinkel}

\subsection{Netzebenenabstand}

\chapter{Bragg-Reflexion}

\section{Aufbau}

\parencite{wikipedia/Goniometer}
\parencite{wikipedia/hygroskopie}
\parencite{leybold/554831}
\parencite{leybold/554800}

Wir stellen den Taster auf \textsc{coupled}, damit der Winkel des Zählrohrs immer
der doppelte Winkel der Probe ist. \parencite{leybold/554800} Auf diese Weise
erhalten wir eine optimale Bragg-Beugung und maximale Intensität.

\section{Bestimmung des Anodenmaterials}

\begin{figure}[htbp]
    \centering
    \includegraphics[width=\linewidth]{Roehre_2.pdf}
    \caption{%
        Spektrum der unbekannten Röhre mit Nummer 2.
    }
    \label{fig:}
\end{figure}

\subsection{Auswertung}

\section{Feinstrukturmessung}

\begin{figure}[htbp]
    \centering
    \includegraphics[width=\linewidth]{Langzeit.pdf}
    \caption{%
        Spektrum der Molybdän-Röhre. Es ist die Aufspaltung der
        $K_\alpha$-Linie in vierter Beugungsordnung zu sehen.
    }
    \label{fig:}
\end{figure}

\subsection{Auswertung}

\chapter{Zerstörungsfreie Aufnahme chemischer Zusammensetzungen}

\section{Aufbau}

\subsection{Tausch der Röntgenröhre}

\subsection{Einbau Energiedetektor}

\section{Kalibrierung mit FeZn}

\subsection{Durchführung}

\subsection{Auswertung}

\begin{figure}[htbp]
    \centering
    \includegraphics[width=\linewidth]{FeZn.pdf}
    \caption{%
        Spektrum von FeZn.
    }
    \label{fig:FeZn}
\end{figure}

\begin{figure}[htbp]
    \centering
    \includegraphics[width=\linewidth]{Energieeichung.pdf}
    \caption{%
        Anpassung der Energie an die Kanalnummer.
    }
    \label{fig:}
\end{figure}

\section{Aufnahme der Legierung}

\subsection{Durchführung}

\subsection{Auswertung}

\begin{figure}[htbp]
    \centering
    %\includegraphics[width=\linewidth]{Probe_1.pdf}
    \caption{%
        Spektrum der unbekannten Probe 1.
    }
    \label{fig:}
\end{figure}

\section{Aufnahme der reinen Metalle}

\subsection{Durchführung}

\subsection{Auswertung}

\section{Bestimmung der Legierung und Massenanteile}

\begin{figure}[htbp]
    \centering
    \includegraphics[width=\linewidth]{Bestimmung.pdf}
    \caption{%
        Spektra der unbekannten Probe sowie der reinen Metalle.
    }
    \label{fig:Bestimmung}
\end{figure}

\begin{figure}[htbp]
    \centering
    \includegraphics[width=\linewidth]{Bestimmung_Zoom.pdf}
    \caption{%
        Ausschnittsvergrößerung von Abbildung~\ref{fig:Bestimmung}
    }
    \label{fig:Bestimmung_Zoom}
\end{figure}

\chapter{Laue-Aufnahme}

\section{Aufbau}

\section{Durchführung}

\subsection{Entwickeln des Films}

\begin{figure}[htbp]
    \centering
    \includegraphics[width=\linewidth]{../Messdaten/laue.png}
    \caption{%
        Scan des Fotos, das wir durch Beugung an einem NaCl-Kristall erzeugt
        haben.
    }
    \label{fig:Laue}
\end{figure}

\section{Auswertung}

\begin{figure}[htbp]
    \centering
    \includegraphics[width=\linewidth]{laue-helligkeit.pdf}
    \caption{%
        Helligkeit im blauen Kanal.
    }
    \label{fig:}
\end{figure}

\begin{figure}[htbp]
    \centering
    \includegraphics[width=\linewidth]{laue-helligkeit-gauss.pdf}
    \caption{%
        Helligkeit nach Gauss'scher Unschärfe mit << gauss_radius >> Pixeln
        Radius.
    }
    \label{fig:}
\end{figure}

\subsection{Ablesen der Reflexpunkte}

\begin{figure}[htbp]
    \centering
    \includegraphics[width=\linewidth]{laue-gradient_x_y.pdf}
    \caption{%
        Amplitude des Gradients
    }
    \label{fig:}
\end{figure}

\begin{figure}[htbp]
    \centering
    \includegraphics[width=\linewidth]{laue-laplace}
    \caption{%
        Laplace
    }
    \label{fig:}
\end{figure}

\begin{figure}[htbp]
    \centering
    \includegraphics[width=\linewidth]{laue-peaks.pdf}
    \caption{%
        Gefundene Maxima
    }
    \label{fig:}
\end{figure}

\subsection{Zuordnung der Gitterebenen}

\chapter{Zusammenfassung}

\chapter{Diskussion}

%%%%%%%%%%%%%%%%%%%%%%%%%%%%%%%%%%%%%%%%%%%%%%%%%%%%%%%%%%%%%%%%%%%%%%%%%%%%%%%
%                                   Anhang                                    %
%%%%%%%%%%%%%%%%%%%%%%%%%%%%%%%%%%%%%%%%%%%%%%%%%%%%%%%%%%%%%%%%%%%%%%%%%%%%%%%

\FloatBarrier
\begin{appendix}
    \FloatBarrier
    \chapter{\LaTeX-Quelltext}

    Der \LaTeX-Quelltext zu allen Protokollen in diesem Praktikum kann auf
    \ref{it:mu} eingesehen werden. Die Quellen für alle Protokolle in diesem
    Praktikum können auf \ref{it:github/alles} eingesehen werden. Die
    \LaTeX-Datei wird aus \ref{it:github/template} generiert.

    \begin{enumerate}
        \item
            \label{it:mu}
            \url{http://martin-ueding.de/de/university/physik412/}
        \item
            \label{it:github/alles}
            \url{https://github.com/martin-ueding/physik412-Protokolle/}
        \item
            \label{it:github/template}
            \url{https://github.com/martin-ueding/physik412-Protokolle/blob/master/\versuchsnummer/Template.tex}
    \end{enumerate}
\end{appendix}

%%%%%%%%%%%%%%%%%%%%%%%%%%%%%%%%%%%%%%%%%%%%%%%%%%%%%%%%%%%%%%%%%%%%%%%%%%%%%%%
%                                  Literatur                                  %
%%%%%%%%%%%%%%%%%%%%%%%%%%%%%%%%%%%%%%%%%%%%%%%%%%%%%%%%%%%%%%%%%%%%%%%%%%%%%%%

\FloatBarrier
\printbibliography

\end{document}

% vim: et spell spelllang=de tw=79
