% Copyright © 2013 Martin Ueding <dev@martin-ueding.de>

\input{../../header.tex}

\usepackage{booktabs}
\usepackage{placeins}

\newcommand\versuchsnummer{428}

\ihead{physik412 – Versuch \versuchsnummer}
\ifoot{Martin Ueding, Lino Lemmer}

\subject{Praktikumsprotokoll}
\title{Röntgenstrahlung und Materialanalyse}
\subtitle{physik412 – Versuch \versuchsnummer}
\author{
    Martin Ueding \\
    \small{\href{mailto:mu@martin-ueding.de}{mu@martin-ueding.de}}
    \and
    Lino Lemmer \\
    \small{\href{mailto:s6lilemm@uni-bonn.de}{s6lilemm@uni-bonn.de}}
}

\setcounter{secnumdepth}{4}
\setcounter{tocdepth}{4}

\begin{document}

\maketitle

\begin{abstract}
    \fehlt
\end{abstract}

\tableofcontents

\chapter{Theorie}

\section{Entstehung der Röntgenstrahlung}

\parencite[Abschnitt~17.3.1]{meschede-gerthsen_24}

\subsection{Bremsstrahlung}

\parencite[Abschnitt~17.3.4]{meschede-gerthsen_24}

\subsection{$K_\alpha$ und weitere Linien}

\parencite[Abschnitt~17.3.5]{meschede-gerthsen_24}

\section{Nachweis von Röntgenstrahlung}

\subsection{Geiger-Müller Zählrohr}

\subsubsection{Aufbau}

\subsubsection{Funktionsweise}

\subsubsection{Totzeit}

\subsection{Röntgenenergiedetektor}

\subsubsection{Aufbau}

\subsubsection{Funktionsweise}

\subsubsection{PIN-Photodiode}

\parencite[Abschnitt~„Photodiode“]{wikipedia/pin-Diode}

\subsubsection{Vielkanalanalysator}

Ein Vielkanalanalysator registriert die Spannungspulse, teilt diese in viele
Spannungsintervalle auf und zählt die Häufigkeit.
\parencite{Phywe/Vierkanalanalysator} So entsteht ein Histogram der Energien.
\parencite{wikipedia/Vielkanalanalysator}

\section{Röntgenfluoreszenz}

\subsection{Bestimmung der Massenanteile}

\section{Bragg-Beugung}

\section{Laue-Verfahren}

\subsection{Laue-Bedingung}

\subsection{Miller'sche Indizes}

\subsection{Elementarzelle}

\subsection{Glanzwinkel}

\subsection{Netzebenenabstand}

\chapter{Bragg-Reflexion}

\section{Aufbau}

\parencite{wikipedia/Goniometer}
\parencite{wikipedia/hygroskopie}
\parencite{leybold/554831}
\parencite{leybold/554800}

Wir stellen den Taster auf \textsc{coupled}, damit der Winkel des Zählrohrs immer
der doppelte Winkel der Probe ist. \parencite{leybold/554800} Auf diese Weise
erhalten wir eine optimale Bragg-Beugung und maximale Intensität.

\section{Bestimmung des Anodenmaterials}

\subsection{Auswertung}

\section{Feinstrukturmessung}

\subsection{Auswertung}

\chapter{Zerstörungsfreie Aufnahme chemischer Zusammensetzungen}

\section{Aufbau}

\subsection{Tausch der Röntgenröhre}

\subsection{Einbau Energiedetektor}

\section{Kalibrierung mit FeZn}

\subsection{Durchführung}

\subsection{Auswertung}

\section{Aufnahme der Legierung}

\subsection{Durchführung}

\subsection{Auswertung}

\section{Aufnahme der reinen Metalle}

\subsection{Durchführung}

\subsection{Auswertung}

\section{Bestimmung der Legierung und Massenanteile}

\chapter{Laue-Aufnahme}

\section{Aufbau}

\section{Durchführung}

\subsection{Entwickeln des Films}

\section{Auswertung}

\subsection{Ablesen der Reflexpunkte}

\subsection{Zuordnung der Gitterebenen}

\chapter{Zusammenfassung}

\chapter{Diskussion}

%%%%%%%%%%%%%%%%%%%%%%%%%%%%%%%%%%%%%%%%%%%%%%%%%%%%%%%%%%%%%%%%%%%%%%%%%%%%%%%
%                                   Anhang                                    %
%%%%%%%%%%%%%%%%%%%%%%%%%%%%%%%%%%%%%%%%%%%%%%%%%%%%%%%%%%%%%%%%%%%%%%%%%%%%%%%

\FloatBarrier
\begin{appendix}
    \FloatBarrier
    \chapter{\LaTeX-Quelltext}

    Der \LaTeX-Quelltext zu allen Protokollen in diesem Praktikum kann auf
    \ref{it:mu} eingesehen werden. Die Quellen für alle Protokolle in diesem
    Praktikum können auf \ref{it:github/alles} eingesehen werden. Die
    \LaTeX-Datei wird aus \ref{it:github/template} generiert.

    \begin{enumerate}
        \item
            \label{it:mu}
            \url{http://martin-ueding.de/de/university/physik412/}
        \item
            \label{it:github/alles}
            \url{https://github.com/martin-ueding/physik412-Protokolle/}
        \item
            \label{it:github/template}
            \url{https://github.com/martin-ueding/physik412-Protokolle/blob/master/\versuchsnummer/Template.tex}
    \end{enumerate}
\end{appendix}

%%%%%%%%%%%%%%%%%%%%%%%%%%%%%%%%%%%%%%%%%%%%%%%%%%%%%%%%%%%%%%%%%%%%%%%%%%%%%%%
%                                  Literatur                                  %
%%%%%%%%%%%%%%%%%%%%%%%%%%%%%%%%%%%%%%%%%%%%%%%%%%%%%%%%%%%%%%%%%%%%%%%%%%%%%%%

\FloatBarrier
\printbibliography

\end{document}

% vim: et spell spelllang=de tw=79
