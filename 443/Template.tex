% Copyright © 2013 Martin Ueding <dev@martin-ueding.de>

\input{../../header.tex}

\usepackage{keystroke}
\usepackage{placeins}
\usepackage{pdflscape}
\usepackage{booktabs}
\usepackage{minted}
\usepackage{multicol}

\newcommand\versuchsnummer{443}

\ihead{physik412 – Versuch \versuchsnummer}
\ifoot{Martin Ueding, Lino Lemmer}

\subject{Praktikumsprotokoll}
\title{Kernmagnetische Relaxation}
\subtitle{physik412 – Versuch \versuchsnummer}
\author{
    Martin Ueding \\
    \small{\href{mailto:mu@martin-ueding.de}{mu@martin-ueding.de}}
    \and
    Lino Lemmer \\
    \small{\href{mailto:s6lilemm@uni-bonn.de}{s6lilemm@uni-bonn.de}}
}

\setcounter{secnumdepth}{4}
\setcounter{tocdepth}{4}

\begin{document}

\maketitle

\begin{abstract}
    \fehlt
\end{abstract}

\tableofcontents

%%%%%%%%%%%%%%%%%%%%%%%%%%%%%%%%%%%%%%%%%%%%%%%%%%%%%%%%%%%%%%%%%%%%%%%%%%%%%%%
%                                   Theorie                                   %
%%%%%%%%%%%%%%%%%%%%%%%%%%%%%%%%%%%%%%%%%%%%%%%%%%%%%%%%%%%%%%%%%%%%%%%%%%%%%%%

\chapter{Theorie}

\cite{teach_spin_manual}

\cite[Abschnitt~15.9.3 „Magnetische~Resonanz”]{meschede-gerthsen_24}

\cite{physik412-Anleitung}

\section{Spin $1/2$ Teilchen im homogenen Magnetfeld}

In diesem Versuch untersuchen wie die magnetische Resonanz von Protonen, die in
Form von Wasserstoff in einer großen Form in der Probe aus leichtem Mineralöl
vorliegen. Die Protonen zeichnen sich durch einen Spin $1/2$ aus.

Dieser Spin wurde in „Experimentalphysik 4“ eingeführt und verhält sich wie ein
Drehimpuls. Die Quantenzahl $s$ ist hier immer $1/2$. Es bleibt die
$z$-Komponente $m_s$, die die Werte $-1/2$ und $1/2$ annehmen kann.

Die Protonen besitzen ein magnetisches Moment $\vec\mu$, das immer parallel zum
Spin ist und eine feste Magnitude besitzt:
\[
    \mu = \gamma \hbar s
\]

Der Faktor $\gamma$ ist das gyromagnetische Verhältnis und ist für verschiedene
Teilchen (z.\,B. Protonen, Elektronen) unterschiedlich.

\subsection{Aufspaltung in zwei Zustände}

In einem homogenen Magnetfeld $\vec B_0$ erfährt das magnetische Moment
$\vec\mu$ ein Drehmoment $\vec M = \vec \mu \times \vec B_0$, das zu einer
Einstellenergie führt:
\[
    E = - \inner{\vec \mu}{\vec B_0}
\]

Da die magnetische Spinquantenzahl $m_s$ nur die Werte $-1/2$ und $1/2$ annehmen
darf, gibt es nur zwei Zustände, $\ket\uparrow$ und $\ket\downarrow$, die
aufgrund der Einstellenergie unterschiedliche Energie haben. Ohne das externe
Magnetfeld wären diese Zustände entartet.

\subsection{Besetzung, thermisches Gleichgewicht}

Ohne das externe Feld liegen alle Spins in einem beliebigen Zustand vor, so
dass es keine makroskopische Magnetisierung $\vec M$ gibt.

\subsection{Longitudinale Relaxation (Relaxationszeit $T_1$)}
\subsection{Zeeman–Effekt (nur kurz)}
\subsection{Präzession um $z$–Achse}

Da das magnetische Moment $\vec\mu$ wahrscheinlich nicht parallel zum
magnetischen Feld ist, wird auf es konstant das Drehmoment durch das
magnetische Feld wirken. Dies führt dazu, dass das magnetische Moment um die
$z$-Achse präzessiert. Die Frequenz dieser Präzession ist die Larmorfrequenz:
\[
    \omega_\text L = \gamma B_0
\]

Die makroskopische Magnetisierung, die sich als Mittel aus allen magnetischen
Momenten ergibt, wird daher auch um die $z$-Achse präzessieren, zumindest
solange die einzelnen Spins mit der gleichen Frequenz präzessieren. Dazu später
mehr.

\section{Reaktion von Spin auf zusätzliches Feld}

In diesem Experiment werden wir zusätzlich zum (möglichst) homogenen Feld $\vec
B_0$ ein gepulstes, sich drehendes, Feld $B_\text{RF}$ verwenden, dessen
Frequenz im Radiobereich (\si{\mega\hertz}) liegt.

\subsection{Blochkugel}

Die Spins haben im externen Feld die Basiszustände $\ket\uparrow$ und
$\ket\downarrow$. Es können aber auch Zustände auftreten, die sich durch zwei
Mischwinkel $\theta$ und $\phi$ charakterisieren lassen:
\parencite{wikipedia/bloch_kugel}
\[
    \ket{\theta, \phi}
    = \cos\del{\frac\theta2} \ket\uparrow
    + \eup^{\iup \phi} \sin\del{\frac\theta2} \ket\downarrow
\]

Durch diese Superposition ist es möglich, dass das magnetische Moment in jede
beliebige Richtung liegen kann. Die energetisch günstigste Variante ist jedoch,
wenn der Spin möglichst parallel zum Magnetfeld $\vec B_0$ liegt, also
$\ket\uparrow$.

\subsection{Rabioszillation}

\cite[Abschnitt~15.9.5 „Rabi-Atomstrahlresonanz”]{meschede-gerthsen_24}

\cite{wikipedia/Rabi_Oszillation}

\subsection{Rotierendes Bezugssystem}
\subsection{$\pi/2$–Puls}

Ein Puls des rotierenden Magnetfeldes der richtigen Stärke und länge wird dazu
führen, dass das magnetische Moment für den Augenblick nur um die mitgeführte
$x$-Achse, also die $x^*$-Achse präzessiert und somit von der $z^*$ auf die
$y^*$-Achse kippt. Ist dies erreicht, muss der Puls beendet sein, damit das
Moment nicht noch weiter kippt.

Das magnetische Moment einer überwiegenden Anzahl Teilchen liegt nun auf der
$x^*$-Achse. Somit liegt auch die Magnetisierung $\vec M$ auf der $x^*$-Achse.
Da sich das Bezugssystem um die $z$-Achse dreht, ist auf der (festen) $x$-Achse
nun eine makroskopische Magnetisierung zu beobachten, die sich zeitlich
sinusförmig verhält.

Mit der Messspule ist nun die mit der Larmorfrequenz oszillierende
Magnetisierung zu beobachten.

Abbildung 15.55 aus Gerthsen Physik von Seite 765

\subsection{$\pi$–Puls}
\subsection{Transversale Relaxation}
\subsubsection{Homogene Transversale Relaxationszeit $T_2$}
\subsubsection{Effektive Transversale Relaxationszeit $T_2^*$}

\section{Messmethoden}
\subsection{Konstanter Spannungsoffset}
\subsection{Longitudinale Relaxationszeit $T_1$}
\subsubsection{Sättigungs–Zurückgewinnung}

Formel P443.1

Formel P443.2

\subsubsection{Polarisations–Zurückgewinnung}

Formel P443.3

Formel P443.4

\subsection{Transversale Relaxationszeit $T_2$}
\subsubsection{FID-Signal}
\subsubsection{Hahn–Spinecho–Sequenz}
\subsubsection{Carr–Purcell–Sequenz}
\subsubsection{Meiboom–Gill–Sequenz}

\chapter{Aufbau und Kalibrierung}
\section{Magnet und PS2 Controller}
\section{Mainframe und Oszilloskop}

\section{Temperatureinstellung}
\section{Tuning des RF-Resonanzkreises}

Wir stellen \texttt{A\_len} auf \SI{2.5}{\micro\second} und \texttt P auf
\SI{100}{\micro\second}.

\texttt{print\_001}

\texttt{print\_002.csv} zeigt die Antwort bei 10 ns/div

\SI{47.4}{\nano\second}

\section{Optimierung des Free Induction Decay Signals}

Wir stellen das Oszilloskop so ein, dass wir das Signal gut sehen können.

Die Gradienten haben wir eingestellt auf: $X = 0.5$, $Y = 18.0$, $Z = 32.2$ und
$Z^2 = 0.0$.

21.16141

003

\section{$\frac\pi2$- und $\pi$-Puls Justage}

Die Pulsdauer, bei der das Signal minimal ist, erhalten wir bei \texttt{A\_len}
= \SI{5.16}{\micro\second}. Die Dauer für das $\piup/2$-Signal ist entsprechend
\SI{2.58}{\micro\second}.

\chapter{Durchführung und Auswertung}
\section{Rabi-Oszillationen}

\begin{figure}[htbp]
    \centering
    \includegraphics[width=\linewidth]{Rabi.pdf}
    \caption{%
        Messdaten und angepasste Funktionen zur Rabi-Oszillation.
    }
    \label{fig:rabi}
\end{figure}

\section{Longitudinale Relaxationszeit $T_1$}
\subsection{Sättigungs-Zurückgewinnung}
\subsection{Polarisations-Zurückgewinnung}

\section{Effektive Transversale Relaxationszeit $T_2^*$}
\subsection{Homogene Transversale Relaxationszeit $T_2$}
\subsubsection{Hahn–Spinecho–Sequenz}
\subsubsection{Carr–Purcell–Sequenz}
\subsubsection{Meiboom–Gill–Sequenz}


%%%%%%%%%%%%%%%%%%%%%%%%%%%%%%%%%%%%%%%%%%%%%%%%%%%%%%%%%%%%%%%%%%%%%%%%%%%%%%%
%                               Zusammenfassung                               %
%%%%%%%%%%%%%%%%%%%%%%%%%%%%%%%%%%%%%%%%%%%%%%%%%%%%%%%%%%%%%%%%%%%%%%%%%%%%%%%

\FloatBarrier
\chapter{Zusammenfassung}

\fehlt

%%%%%%%%%%%%%%%%%%%%%%%%%%%%%%%%%%%%%%%%%%%%%%%%%%%%%%%%%%%%%%%%%%%%%%%%%%%%%%%
%                                   Anhang                                    %
%%%%%%%%%%%%%%%%%%%%%%%%%%%%%%%%%%%%%%%%%%%%%%%%%%%%%%%%%%%%%%%%%%%%%%%%%%%%%%%

\FloatBarrier
\begin{appendix}
    \FloatBarrier
    \chapter{\LaTeX-Quelltext}

    Der \LaTeX-Quelltext zu allen Protokollen in diesem Praktikum kann auf
    \ref{it:mu} eingesehen werden. Die Quellen für alle Protokolle in diesem
    Praktikum können auf \ref{it:github/alles} eingesehen werden. Die
    \LaTeX-Datei wird aus \ref{it:github/template} generiert.

    \begin{enumerate}
        \item
            \label{it:mu}
            \url{http://martin-ueding.de/de/university/physik412/}
        \item
            \label{it:github/alles}
            \url{https://github.com/martin-ueding/physik412-Protokolle/}
        \item
            \label{it:github/template}
            \url{https://github.com/martin-ueding/physik412-Protokolle/blob/master/\versuchsnummer/Template.tex}
    \end{enumerate}
\end{appendix}

%%%%%%%%%%%%%%%%%%%%%%%%%%%%%%%%%%%%%%%%%%%%%%%%%%%%%%%%%%%%%%%%%%%%%%%%%%%%%%%
%                                  Literatur                                  %
%%%%%%%%%%%%%%%%%%%%%%%%%%%%%%%%%%%%%%%%%%%%%%%%%%%%%%%%%%%%%%%%%%%%%%%%%%%%%%%

\FloatBarrier
\printbibliography

\end{document}

% vim: et spell spelllang=de tw=79
