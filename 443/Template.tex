% Copyright © 2013 Martin Ueding <dev@martin-ueding.de>

\input{../../header.tex}

\usepackage{keystroke}
\usepackage{placeins}
\usepackage{pdflscape}
\usepackage{booktabs}
\usepackage{minted}
\usepackage{multicol}

\newcommand\versuchsnummer{443}

\ihead{physik412 – Versuch \versuchsnummer}
\ifoot{Martin Ueding, Lino Lemmer}

\subject{Praktikumsprotokoll}
\title{Kernmagnetische Relaxation}
\subtitle{physik412 – Versuch \versuchsnummer}
\author{
    Martin Ueding \\
    \small{\href{mailto:mu@martin-ueding.de}{mu@martin-ueding.de}}
    \and
    Lino Lemmer \\
    \small{\href{mailto:s6lilemm@uni-bonn.de}{s6lilemm@uni-bonn.de}}
}

\begin{document}

\maketitle

\begin{abstract}
\end{abstract}

\tableofcontents

Quellen, die wir schon zur Vorbereitung gelesen haben:

\cite{teach_spin_manual}

\cite[Abschnitt~15.9.3]{meschede-gerthsen_24}

\cite{physik412-Anleitung}

Struktur, die mir (Martin) sinnvoll erscheint:

\begin{verbatim}
– Theorie
  – Spin 1/2 Teilchen im homogenen Magnetfeld
    – Aufspaltung in zwei Zustände
    – Besetzung, thermisches Gleichgewicht
    – Longitudinale Relaxation (Relaxationszeit T_1)
    – Zeeman–Effekt (nur kurz)
    – Präzession um z–Achse
  – Reaktion von Spin auf zusätzliches Feld
    – Rotierendes Bezugssystem
    – \pi/2–Puls
    – \pi–Puls
    – Transversale Relaxation
      - Homogene Transversale Relaxationszeit T_2
      - Effektive Transversale Relaxationszeit T_2^*
  – Messmethoden
    – Longitudinale Relaxationszeit T_1
      – Sättigungs–Zurückgewinnung
      – Polarisations–Zurückgewinnung
    – Transversale Relaxationszeit T_2
      – Hahn–Spinecho–Sequenz
      – Carr–Purcell–Sequenz
      – Meiboom–Gill–Sequenz
\end{verbatim}

%%%%%%%%%%%%%%%%%%%%%%%%%%%%%%%%%%%%%%%%%%%%%%%%%%%%%%%%%%%%%%%%%%%%%%%%%%%%%%%
%                                   Theorie                                   %
%%%%%%%%%%%%%%%%%%%%%%%%%%%%%%%%%%%%%%%%%%%%%%%%%%%%%%%%%%%%%%%%%%%%%%%%%%%%%%%

\FloatBarrier
\section{Theorie}

%%%%%%%%%%%%%%%%%%%%%%%%%%%%%%%%%%%%%%%%%%%%%%%%%%%%%%%%%%%%%%%%%%%%%%%%%%%%%%%
%                               Zusammenfassung                               %
%%%%%%%%%%%%%%%%%%%%%%%%%%%%%%%%%%%%%%%%%%%%%%%%%%%%%%%%%%%%%%%%%%%%%%%%%%%%%%%

\FloatBarrier
\section{Zusammenfassung}

\fehlt

%%%%%%%%%%%%%%%%%%%%%%%%%%%%%%%%%%%%%%%%%%%%%%%%%%%%%%%%%%%%%%%%%%%%%%%%%%%%%%%
%                                   Anhang                                    %
%%%%%%%%%%%%%%%%%%%%%%%%%%%%%%%%%%%%%%%%%%%%%%%%%%%%%%%%%%%%%%%%%%%%%%%%%%%%%%%

\FloatBarrier
\begin{appendix}
    \FloatBarrier
    \section{\LaTeX-Quelltext}

    Der \LaTeX-Quelltext zu allen Protokollen in diesem Praktikum kann auf
    \ref{it:mu} eingesehen werden. Die Quellen für alle Protokolle in diesem
    Praktikum können auf \ref{it:github/alles} eingesehen werden. Die
    \LaTeX-Datei wird aus \ref{it:github/template} generiert.

    \begin{enumerate}
        \item
            \label{it:mu}
            \url{http://martin-ueding.de/de/university/physik412/}
        \item
            \label{it:github/alles}
            \url{https://github.com/martin-ueding/physik412-Protokolle/}
        \item
            \label{it:github/template}
            \url{https://github.com/martin-ueding/physik412-Protokolle/blob/master/\versuchsnummer/Template.tex}
    \end{enumerate}
\end{appendix}

%%%%%%%%%%%%%%%%%%%%%%%%%%%%%%%%%%%%%%%%%%%%%%%%%%%%%%%%%%%%%%%%%%%%%%%%%%%%%%%
%                                  Literatur                                  %
%%%%%%%%%%%%%%%%%%%%%%%%%%%%%%%%%%%%%%%%%%%%%%%%%%%%%%%%%%%%%%%%%%%%%%%%%%%%%%%

\FloatBarrier
\printbibliography

\end{document}

% vim: et spell spelllang=de tw=79
