% Copyright © 2013 Martin Ueding <dev@martin-ueding.de>

\input{../../header.tex}

\usepackage{keystroke}
\usepackage{placeins}
\usepackage{pdflscape}
\usepackage{booktabs}
\usepackage{minted}
\usepackage{multicol}

\newcommand\versuchsnummer{443}

\ihead{physik412 – Versuch \versuchsnummer}
\ifoot{Martin Ueding, Lino Lemmer}

\subject{Praktikumsprotokoll}
\title{Kernmagnetische Relaxation}
\subtitle{physik412 – Versuch \versuchsnummer}
\author{
    Martin Ueding \\
    \small{\href{mailto:mu@martin-ueding.de}{mu@martin-ueding.de}}
    \and
    Lino Lemmer \\
    \small{\href{mailto:s6lilemm@uni-bonn.de}{s6lilemm@uni-bonn.de}}
}

\setcounter{secnumdepth}{4}
\setcounter{tocdepth}{4}

\begin{document}

\maketitle

\begin{abstract}
    \fehlt
\end{abstract}

\tableofcontents

%%%%%%%%%%%%%%%%%%%%%%%%%%%%%%%%%%%%%%%%%%%%%%%%%%%%%%%%%%%%%%%%%%%%%%%%%%%%%%%
%                                   Theorie                                   %
%%%%%%%%%%%%%%%%%%%%%%%%%%%%%%%%%%%%%%%%%%%%%%%%%%%%%%%%%%%%%%%%%%%%%%%%%%%%%%%

\chapter{Theorie}

\cite{teach_spin_manual}

\cite[Abschnitt~15.9.3 „Magnetische~Resonanz”]{meschede-gerthsen_24}

\cite{physik412-Anleitung}

\section{Spin $1/2$ Teilchen im homogenen Magnetfeld}
\subsection{Aufspaltung in zwei Zustände}
\subsection{Besetzung, thermisches Gleichgewicht}
\subsection{Longitudinale Relaxation (Relaxationszeit $T_1$)}
\subsection{Zeeman–Effekt (nur kurz)}
\subsection{Präzession um $z$–Achse}

\section{Reaktion von Spin auf zusätzliches Feld}
\subsection{Blochkugel}
\subsection{Rabioszillation}

\cite[Abschnitt~15.9.5 „Rabi-Atomstrahlresonanz”]{meschede-gerthsen_24}

\cite{wikipedia/Rabi_Oszillation}

\subsection{Rotierendes Bezugssystem}
\subsection{$\pi/2$–Puls}

Abbildung 15.55 aus Gerthsen Physik von Seite 765

\subsection{$\pi$–Puls}
\subsection{Transversale Relaxation}
\subsubsection{Homogene Transversale Relaxationszeit $T_2$}
\subsubsection{Effektive Transversale Relaxationszeit $T_2^*$}

\section{Messmethoden}
\subsection{Konstanter Spannungsoffset}
\subsection{Longitudinale Relaxationszeit $T_1$}
\subsubsection{Sättigungs–Zurückgewinnung}

Formel P443.1

Formel P443.2

\subsubsection{Polarisations–Zurückgewinnung}

Formel P443.3

Formel P443.4

\subsection{Transversale Relaxationszeit $T_2$}
\subsubsection{FID-Signal}
\subsubsection{Hahn–Spinecho–Sequenz}
\subsubsection{Carr–Purcell–Sequenz}
\subsubsection{Meiboom–Gill–Sequenz}

\chapter{Aufbau und Kalibrierung}
\section{Magnet und PS2 Controller}
\section{Mainframe und Oszilloskop}

\section{Temperatureinstellung}
\section{Tuning des RF-Resonanzkreises}

Wir stellen \texttt{A_len} auf \SI{2.5}{\micro\second} und \texttt P auf
\SI{100}{\micro\second}.

print_001

\texttt{print_002.csv} zeigt die Antwort bei 10 ns/div

\SI{47.4}{\nano\second}

\section{Optimierung des Free Induction Decay Signals}

Wir stellen das Oszilloskop so ein, dass wir das Signal gut sehen können.

Die Gradienten haben wir eingestellt auf: $X = 0.5$, $Y = 18.0$, $Z = 32.2$ und
$Z^2 = 0.0$.

21.16141

003

\section{$\frac\pi2$- und $\pi$-Puls Justage}

Die Pulsdauer, bei der das Signal minimal ist, erhalten wir bei \texttt{A_len}
= \SI{5.16}{\micro\second}. Die Dauer für das $\piup/2$-Signal ist entsprechend
\SI{2.58}{\micro\second}.

\chapter{Durchführung und Auswertung}
\section{Rabi-Oszillationen}

\section{Longitudinale Relaxationszeit $T_1$}
\subsection{Sättigungs-Zurückgewinnung}
\subsection{Polarisations-Zurückgewinnung}

\section{Effektive Transversale Relaxationszeit $T_2^*$}
\subsection{Homogene Transversale Relaxationszeit $T_2$}
\subsubsection{Hahn–Spinecho–Sequenz}
\subsubsection{Carr–Purcell–Sequenz}
\subsubsection{Meiboom–Gill–Sequenz}


%%%%%%%%%%%%%%%%%%%%%%%%%%%%%%%%%%%%%%%%%%%%%%%%%%%%%%%%%%%%%%%%%%%%%%%%%%%%%%%
%                               Zusammenfassung                               %
%%%%%%%%%%%%%%%%%%%%%%%%%%%%%%%%%%%%%%%%%%%%%%%%%%%%%%%%%%%%%%%%%%%%%%%%%%%%%%%

\FloatBarrier
\chapter{Zusammenfassung}

\fehlt

%%%%%%%%%%%%%%%%%%%%%%%%%%%%%%%%%%%%%%%%%%%%%%%%%%%%%%%%%%%%%%%%%%%%%%%%%%%%%%%
%                                   Anhang                                    %
%%%%%%%%%%%%%%%%%%%%%%%%%%%%%%%%%%%%%%%%%%%%%%%%%%%%%%%%%%%%%%%%%%%%%%%%%%%%%%%

\FloatBarrier
\begin{appendix}
    \FloatBarrier
    \chapter{\LaTeX-Quelltext}

    Der \LaTeX-Quelltext zu allen Protokollen in diesem Praktikum kann auf
    \ref{it:mu} eingesehen werden. Die Quellen für alle Protokolle in diesem
    Praktikum können auf \ref{it:github/alles} eingesehen werden. Die
    \LaTeX-Datei wird aus \ref{it:github/template} generiert.

    \begin{enumerate}
        \item
            \label{it:mu}
            \url{http://martin-ueding.de/de/university/physik412/}
        \item
            \label{it:github/alles}
            \url{https://github.com/martin-ueding/physik412-Protokolle/}
        \item
            \label{it:github/template}
            \url{https://github.com/martin-ueding/physik412-Protokolle/blob/master/\versuchsnummer/Template.tex}
    \end{enumerate}
\end{appendix}

%%%%%%%%%%%%%%%%%%%%%%%%%%%%%%%%%%%%%%%%%%%%%%%%%%%%%%%%%%%%%%%%%%%%%%%%%%%%%%%
%                                  Literatur                                  %
%%%%%%%%%%%%%%%%%%%%%%%%%%%%%%%%%%%%%%%%%%%%%%%%%%%%%%%%%%%%%%%%%%%%%%%%%%%%%%%

\FloatBarrier
\printbibliography

\end{document}

% vim: et spell spelllang=de tw=79
