% Copyright © 2013 Martin Ueding <dev@martin-ueding.de>

\input{../../header.tex}

\usepackage[section]{placeins}
\usepackage{booktabs}

\newcommand\versuchsnummer{424}

\ihead{physik412 – Versuch \versuchsnummer}
\ifoot{M. Ueding, L. Lemmer, G. Ahmad}

\subject{Praktikumsprotokoll}
\title{Halleffekt in Halbleitern}
\subtitle{physik412 – Versuch \versuchsnummer}
\author{
    Martin Ueding \\
    \small{\href{mailto:mu@martin-ueding.de}{mu@martin-ueding.de}}
    \and
    Lino Lemmer \\
    \small{\href{mailto:s6lilemm@uni-bonn.de}{s6lilemm@uni-bonn.de}}
    \and
    Goran Ahmad \\
    \small{\href{mailto:s6goahma@uni-bonn.de}{s6goahma@uni-bonn.de}}
}
\publishers{Tutor: Christian Hammann}

\setcounter{secnumdepth}{4}
\setcounter{tocdepth}{3}

\begin{document}

\maketitle

\begin{abstract}
    \fehlt
\end{abstract}

\tableofcontents


\newcommand\probeA{InAs~HF-540}
\newcommand\probeB{InAs~HF-301-040}

\chapter{Einleitung}

\FloatBarrier
\chapter{Theorie}

\FloatBarrier
\chapter{Aufbau}

\FloatBarrier
\chapter{Durchführung}

\FloatBarrier
\section{Bestimmung des spezifischen Widerstands}

$I = \SI{13.601}{\milli\ampere}$

\FloatBarrier
\subsection{\probeA}

\begin{table}[htbp]
    \centering
    \begin{tabular}{SSSSSS}
        {$R_{1234} / \si\ohm$} &
        {$R_{2341} / \si\ohm$} &
        {$R_{3412} / \si\ohm$} &
        {$R_{4132} / \si\ohm$} &
        {$R_{1234} / R_{2341}$} &
        {$R_{3412} / R_{4121}$} \\
        \midrule
        %< for row in hf_540_strom_R_tabelle: ->%
        << row | join(' & ') >> \\
        %< endfor ->%
    \end{tabular}
    \caption{%
        Widerstände für die Probe \probeA.
    }
    \label{tab:Ar}
\end{table}

\begin{table}[htbp]
    \centering
    \begin{tabular}{SSSSS}
        {$\rho_1 d^{-1} / \si\ohm$} &
        {$\rho_2 d^{-1} / \si\ohm$} &
        {$\rho d^{-1} / \si\ohm$} \\
        \midrule
        %< for row in hf_540_strom_rho_tabelle: ->%
        << row | join(' & ') >> \\
        %< endfor ->%
    \end{tabular}
    \caption{%
        Spezifische Widerstände für die Probe \probeA.
    }
    \label{tab:Arho}
\end{table}

Wir bilden Mittelwert und Standardabweichung und erhalten so $\rho = \SI{<<
hf_540_strom_rho >>}{\ohm}$ aus allen fünf Messreihen.

\FloatBarrier
\subsection{\probeB}

\begin{table}[htbp]
    \centering
    \begin{tabular}{SSSSSS}
        {$R_{1234} / \si\ohm$} &
        {$R_{2341} / \si\ohm$} &
        {$R_{3412} / \si\ohm$} &
        {$R_{4132} / \si\ohm$} &
        {$R_{1234} / R_{2341}$} &
        {$R_{3412} / R_{4121}$} \\
        \midrule
        %< for row in hf_301_strom_R_tabelle: ->%
        << row | join(' & ') >> \\
        %< endfor ->%
    \end{tabular}
    \caption{%
        Widerstände für die Probe \probeB.
    }
    \label{tab:Br}
\end{table}

\begin{table}[htbp]
    \centering
    \begin{tabular}{SSSSS}
        {$\rho_1 d^{-1} / \si\ohm$} &
        {$\rho_2 d^{-1} / \si\ohm$} &
        {$\rho d^{-1} / \si\ohm$} \\
        \midrule
        %< for row in hf_301_strom_rho_tabelle: ->%
        << row | join(' & ') >> \\
        %< endfor ->%
    \end{tabular}
    \caption{%
        Spezifische Widerstände für die Probe \probeA.
    }
    \label{tab:Brho}
\end{table}

Wir bilden Mittelwert und Standardabweichung und erhalten so $\rho = \SI{<<
hf_301_strom_rho >>}{\ohm}$ aus allen fünf Messreihen.

\FloatBarrier
\section{Bestimmung der Hallkonstanten}

$B = \SI{0.138 +- 0.001}{\tesla}$

\FloatBarrier
\subsection{Mit Nullfeldmessung}

\FloatBarrier
\subsubsection{\probeA}

\begin{table}[htbp]
    \centering
    \begin{tabular}{SSSS}
        {$\Deltaup V_\text{A} / \si\volt$} &
        {$\Deltaup V_\text{B} / \si\volt$} &
        {$\Deltaup V_\text{C} / \si\volt$} &
        {$\Deltaup V_\text{D} / \si\volt$} \\
        \midrule
        %< for row in hf_540_hall_V_tabelle: ->%
        << row | join(' & ') >> \\
        %< endfor ->%
    \end{tabular}
    \caption{%
        Spannungsdifferenzen bei der Messung der Hallkonstanten für die Probe
        \probeA.
    }
    \label{tab:AV}
\end{table}

\begin{table}[htbp]
    \centering
    \begin{tabular}{SSSS}
        {$V_{H+} / \si\volt$} &
        {$V_{H-} / \si\volt$} &
        {$R_{H+} d^{-1} / \si{\volt\per\tesla\per\ampere}$} &
        {$R_{H-} d^{-1} / \si{\volt\per\tesla\per\ampere}$} \\
        \midrule
        %< for row in hf_540_hall_V_R_tabelle: ->%
        << row | join(' & ') >> \\
        %< endfor ->%
    \end{tabular}
    \caption{%
        Hallkonstanten für die Probe \probeA.
    }
    \label{tab:AR}
\end{table}

Wir bilden Mittelwert und Standardabweichung und erhalten so $R_{H+} = \SI{<<
hf_540_hall_R_H_plus >>}{\volt\per\tesla\per\ampere}$ und $R_{H-} = \SI{<<
hf_540_hall_R_H_minus >>}{\volt\per\tesla\per\ampere}$ aus allen fünf Messreihen.


\FloatBarrier
\subsubsection{\probeB}


\begin{table}[htbp]
    \centering
    \begin{tabular}{SSSS}
        {$\Deltaup V_\text{A} / \si\volt$} &
        {$\Deltaup V_\text{B} / \si\volt$} &
        {$\Deltaup V_\text{C} / \si\volt$} &
        {$\Deltaup V_\text{D} / \si\volt$} \\
        \midrule
        %< for row in hf_301_hall_V_tabelle: ->%
        << row | join(' & ') >> \\
        %< endfor ->%
    \end{tabular}
    \caption{%
        Spannungsdifferenzen bei der Messung der Hallkonstanten für die Probe
        \probeB.
    }
    \label{tab:BV}
\end{table}

\begin{table}[htbp]
    \centering
    \begin{tabular}{SSSS}
        {$V_{H+} / \si\volt$} &
        {$V_{H-} / \si\volt$} &
        {$R_{H+} d^{-1} / \si{\volt\per\tesla\per\ampere}$} &
        {$R_{H-} d^{-1} / \si{\volt\per\tesla\per\ampere}$} \\
        \midrule
        %< for row in hf_301_hall_V_R_tabelle: ->%
        << row | join(' & ') >> \\
        %< endfor ->%
    \end{tabular}
    \caption{%
        Hallkonstanten für die Probe \probeB.
    }
    \label{tab:BH}
\end{table}

Wir bilden Mittelwert und Standardabweichung und erhalten so $R_{H+} = \SI{<<
hf_301_hall_R_H_plus >>}{\volt\per\tesla\per\ampere}$ und $R_{H-} = \SI{<<
hf_301_hall_R_H_minus >>}{\volt\per\tesla\per\ampere}$ aus allen fünf Messreihen.

\subsection{Ohne Nullfeldmessung}

\subsubsection{\probeA}

\begin{table}[htbp]
    \centering
    \begin{tabular}{SSS}
        {$R_{H1} d^{-1} / \si{\volt\per\tesla\per\ampere}$} &
        {$R_{H2} d^{-1} / \si{\volt\per\tesla\per\ampere}$} &
        {$R_{H} d^{-1} / \si{\volt\per\tesla\per\ampere}$} \\
        \midrule
        %< for row in hf_540_hall_RH_mit_tabelle: ->%
        << row | join(' & ') >> \\
        %< endfor ->%
    \end{tabular}
    \caption{%
        Hallkonstanten für die Probe \probeA, nach der Auswertungsmethode ohne
        Nullmessung.
    }
    \label{tab:AH_ohne}
\end{table}

Wir bilden Mittelwert und Standardabweichung und erhalten so $R_{H} = \SI{<<
hf_540_hall_R_H >>}{\volt\per\tesla\per\ampere}$ aus allen fünf Messreihen.


\subsubsection{\probeB}

\begin{table}[htbp]
    \centering
    \begin{tabular}{SSS}
        {$R_{H1} d^{-1} / \si{\volt\per\tesla\per\ampere}$} &
        {$R_{H2} d^{-1} / \si{\volt\per\tesla\per\ampere}$} &
        {$R_{H} d^{-1} / \si{\volt\per\tesla\per\ampere}$} \\
        \midrule
        %< for row in hf_301_hall_RH_mit_tabelle: ->%
        << row | join(' & ') >> \\
        %< endfor ->%
    \end{tabular}
    \caption{%
        Hallkonstanten für die Probe \probeB, nach der Auswertungsmethode ohne
        Nullmessung.
    }
    \label{tab:BH_ohne}
\end{table}

Wir bilden Mittelwert und Standardabweichung und erhalten so $R_{H} = \SI{<<
hf_301_hall_R_H >>}{\volt\per\tesla\per\ampere}$ aus allen fünf Messreihen.


\subsection{Diskussion}

In Tabelle~\ref{tab:zimmertemperatur_tabelle1} haben wir die endgültigen Werte
für $\rho$ zusammengetragen und $\mu_\text H$ sowie $p$ berechnet. In
Tabelle~\ref{tab:zimmertemperatur_tabelle2} sind die Hallkonstanten
zusammengefasst.

\begin{table}[htbp]
    \centering
    \begin{tabular}{lSSS}
        Probe &
        {$\rho d^{-1} / \si{\ohm}$} &
        {$\mu_{\text H} d^{-1} / \si{\ohm\volt\per\tesla\per\ampere}$} &
        {$p / \SI{e15}{\per\cubic\meter}$} \\
        \midrule
        %< for row in zimmertemperatur_tabelle1: ->%
        << row | join(' & ') >> \\
        %< endfor ->%
    \end{tabular}
    \caption{%
        Zusammenstellung der Ergebnisse aus dem ersten Versuchsteil, Teil~1.
    }
    \label{tab:zimmertemperatur_tabelle1}
\end{table}

\begin{table}[htbp]
    \centering
    \begin{tabular}{lSSS}
        Probe &
        {$R_{\text H+} d^{-1} / \si{\volt\per\tesla\per\ampere}$} &
        {$R_{\text H-} d^{-1} / \si{\volt\per\tesla\per\ampere}$} &
        {$R_{\text H} d^{-1} / \si{\volt\per\tesla\per\ampere}$} \\
        \midrule
        %< for row in zimmertemperatur_tabelle2: ->%
        << row | join(' & ') >> \\
        %< endfor ->%
    \end{tabular}
    \caption{%
        Zusammenstellung der Ergebnisse aus dem ersten Versuchsteil, Teil~2.
    }
    \label{tab:zimmertemperatur_tabelle2}
\end{table}

Es fällt auf, dass die Leitfähigkeiten recht nahe beiander
liegen. Die Hallkonstanten liegen betragsmäßig ebenfalls recht nahe
beieinander, unterscheiden sich jedoch im Vorzeichen.

In beiden Fällen ist $R_\text H$ innerhalb des Fehlerbereichs der Mittelwert
auf $H_{\text H+}$ und $H_{\text H-}$. Dies bedeutet, dass beide Mess- und
Auswertungsmethoden das Nullfeld gleich korrigieren.

Aus der Ladungsträgerkonzentration (siehe
Tabelle~\ref{tab:zimmertemperatur_tabelle1}), die für positive Ladungsträger
berechnet worden ist, folgt, dass die Probe~\probeA{} negativ und die
Probe~\probeB{} positiv dotiert ist.

\begin{table}[htbp]
    \centering
    \begin{tabular}{SSSSS}
        {$T$} &
        {$\rho$} &
        {$R_\text H$} &
        {$p$} &
        {$\mu$} \\
        \midrule
        %< for row in temperatur_tabelle: ->%
        << row | join(' & ') >> \\
        %< endfor ->%
    \end{tabular}
    \caption{%
    }
    \label{tab:temperatur_tabelle}
\end{table}

\begin{figure}[htbp]
    \centering
    \includegraphics[width=\linewidth]{plot1.pdf}
    \caption{%
    }
    \label{fig:}
\end{figure}

\begin{figure}[htbp]
    \centering
    \includegraphics[width=\linewidth]{plot2.pdf}
    \caption{%
    }
    \label{fig:}
\end{figure}

\begin{figure}[htbp]
    \centering
    \includegraphics[width=\linewidth]{plot3.pdf}
    \caption{%
    }
    \label{fig:}
\end{figure}

%%%%%%%%%%%%%%%%%%%%%%%%%%%%%%%%%%%%%%%%%%%%%%%%%%%%%%%%%%%%%%%%%%%%%%%%%%%%%%%
%                                   Anhang                                    %
%%%%%%%%%%%%%%%%%%%%%%%%%%%%%%%%%%%%%%%%%%%%%%%%%%%%%%%%%%%%%%%%%%%%%%%%%%%%%%%

\FloatBarrier
\begin{appendix}
    \FloatBarrier
    \chapter{\LaTeX-Quelltext}

    Der \LaTeX-Quelltext zu allen Protokollen in diesem Praktikum kann auf
    \ref{it:mu} eingesehen werden. Die Quellen für alle Protokolle in diesem
    Praktikum können auf \ref{it:github/alles} eingesehen werden. Die
    \LaTeX-Datei wird aus \ref{it:github/template} generiert.

    \begin{enumerate}
        \item
            \label{it:mu}
            \url{http://martin-ueding.de/de/university/physik412/}
        \item
            \label{it:github/alles}
            \url{https://github.com/martin-ueding/physik412-Protokolle/}
        \item
            \label{it:github/template}
            \url{https://github.com/martin-ueding/physik412-Protokolle/blob/master/\versuchsnummer/Template.tex}
    \end{enumerate}
\end{appendix}

%%%%%%%%%%%%%%%%%%%%%%%%%%%%%%%%%%%%%%%%%%%%%%%%%%%%%%%%%%%%%%%%%%%%%%%%%%%%%%%
%                                  Literatur                                  %
%%%%%%%%%%%%%%%%%%%%%%%%%%%%%%%%%%%%%%%%%%%%%%%%%%%%%%%%%%%%%%%%%%%%%%%%%%%%%%%

\FloatBarrier
\printbibliography

\end{document}

% vim: et spell spelllang=de tw=79
