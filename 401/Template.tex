% Copyright © 2013 Martin Ueding <dev@martin-ueding.de>

\input{../../header.tex}

\usepackage{csquotes}
\usepackage[section]{placeins}
\usepackage{booktabs}
\usepackage{pdflscape}

\newcommand\versuchsnummer{401}
\DeclareMathOperator\std{std}

\newcommand\erklaerungFehlerNotation{%
    In dieser Notation bedeutet \num{1.234 +- 0.005}, dass der Wert
    $\num{1.234} \pm \num{0.005}$ ist. Die Ziffern in Klammern sind die
    Fehlerangabe. Um den Fehler zu erhalten, wird diese von rechts über die
    Zahl gelegt, alle anderen Stellen werden auf 0 gesetzt.
}

\ihead{physik412 – Versuch \versuchsnummer}
\ifoot{M. Ueding, L. Lemmer}

\subject{Praktikumsprotokoll}
\title{Elektronische Übergänge in Atomen}
\subtitle{physik412 – Versuch \versuchsnummer}
\author{
Martin Ueding \\
\small{\href{mailto:mu@martin-ueding.de}{mu@martin-ueding.de}}
\and
Lino Lemmer \\
\small{\href{mailto:lino.lemmer@uni-bonn.de}{s6lilemm@uni-bonn.de}}
}
\publishers{Tutor: Jens Peter}

\setcounter{secnumdepth}{4}
\setcounter{tocdepth}{3}

\begin{document}

\maketitle

\begin{abstract}
    Im ersten Teil dieses Versuchs betrachten wir mit einem Fabry-Pérot-Etalon
    die, durch den Zeeman-Effekt entstehende, Aufspaltung einer Spektrallinie
    von Cadmium. Aus dieser bestimmen wir das Bohrsche Magneton.

    Im zweiten Teil führen wir den Frank-Hertz-Versuch durch und ermitteln aus
    diesem die Energie des ${}^1\text{D}_2\to{}^1\text{P}_1$-Übergangs von
    Quecksilber.
\end{abstract}

\tableofcontents

\chapter{Theorie}

\section{Energieniveaus von Elektronen im Atom}

\subsection{Ohne äußeres magnetisches Feld}

Löst man die Schrödingergleichung für ein Elektron im radialsymmetrischen
Potenzial des Atomkerns, erhält man aus dieser diskrete Werte für die Energie,
die von der Hauptquantenzahl $n\in\mathbb{N}$ und der Drehimpulsquantenzahl
$l\in\mathbb{N}^0$ abhängen. Für jedes $n$ gibt es $n-1$ mögliche Werte für
$l$. Ist eine $z$-Richtung vorgegeben, so ist die entsprechende Komponente des
Drehimpulses ebenfalls gequantelt, sie beträgt $L_z = \hbar m_l$, wobei $m_l
= -l,\dots,+l$.

Ist kein äußeres Feld vorgegeben, kann der Drehimpuls in beliebige Richtungen
zeigen, ohne dass sich die Energie des Elektrons ändert. Jeder Energiewert ist
also $2l+1$-fach entartet.

\subsection{Mit äußerem magnetischem Feld}

Durch den Drehimpuls erzeugt das Elektron ein magnetisches Moment
\begin{align*}
    \vec\mu &= -\frac{e}{2m_\text{e}}\vec L.
    \intertext{%
        In einem äußeren magnetischen Feld $\vec B=B \vec e_z$ ändert sich
        entsprechend die Energie
    }
    E_\text{mag} &= \skp{\vec\mu}{\vec B} \\
             &= -\frac{e}{2m_\text{e}}\skp{\vec L}{\vec B} \\
             &= -\frac{e}{2m_\text{e}}L_zB \\
             &= -\mu_\text{B}m_lB,
\end{align*}
wobei $\mu_\text{B}$ Bohrsches Magneton heißt. Die Aufspaltung der
Energie, also die Aufhebung der $m_l$-Entartung, heißt Zeeman-Effekt.
Kann der Gesamtspin vernachlässigt werden, spricht man vom normalen
Zeeman-Effekt, ist dies nicht so, muss eine Spin-Bahn-Kopplung beachtet
werden. Die Rechnung dazu funktioniert analog zur obigen Betrachtung,
führt nur zu einer anderen Energieaufspaltung.

\subsection{Niveau-Übergänge}

Elektronen können unter Absorption bzw. Emission von Photonen das Energieniveau
wechseln. Dabei müssen bestimmte Auswahlregeln beachtet werden. Für uns wichtig
sind die Regeln $\Delta l = 1$ und $\Delta m = 0,\pm 1$. Die entsprechenden
Übergänge zwischen $l = 2$ und $l=1$ sind in Abbildung~\ref{fig:Übergänge}
dargestellt. Die Energie der Photonen, die bei gleichgefärbten Übergängen frei
werden sind gleich, sodass hier eine Aufspaltung in drei Linien geschieht.

\begin{figure}
    \centering
    %\includegraphics[width=0.7\textwidth]{../Abbildungen/Übergänge.pdf}
    \caption{%
        Mögliche Übergänge zwischen $l=2$ und $l=1$.
    }
    \label{fig:Übergänge}
\end{figure}


\section{Natürliche Linienbreite und Linienverbreiterung}

\section{Fabry-Pérot-Etalon}

\section{Polarisationsfilter und $\frac{\lambda}4$-Plättchen}

\section{Hall-Sonde}

\section{Frank-Hertz-Versuch}

\subsection{Aufbau und Funktionsweise}

\subsection{Termschema von Quecksilber}

\chapter{Aufbau}

\chapter{Zeeman-Effekt}

\section{Durchführung}

Der Strom, bei der in transversaler Konfiguration die Linien zu unterscheiden
sind, ist \SI{1.9}{\ampere}. In longitudinaler Konfiguration ist der Strom
\SI{1.0}{\ampere}. Die longitudinalen Linien sind deutlich dunkler.

\section{Auswertung}

\begin{figure}[htbp]
    \centering
    \includegraphics[width=\linewidth]{Alles.pdf}
    \caption{%
    }
    \label{fig:}
\end{figure}

\begin{figure}[htbp]
    \centering
    \includegraphics[width=\linewidth]{Zoom.pdf}
    \caption{%
    }
    \label{fig:}
\end{figure}

\section{Diskussion}

\chapter{Frank-Hertz-Versuch}

\section{Durchführung}

\section{Auswertung}

\section{Diskussion}

\printbibliography

\end{document}

% vim: et spell spelllang=de tw=79
