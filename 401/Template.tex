% Copyright © 2013 Martin Ueding <dev@martin-ueding.de>

\input{../../header.tex}

\usepackage{csquotes}
\usepackage[section]{placeins}
\usepackage{booktabs}
\usepackage{pdflscape}

\newcommand\versuchsnummer{402}
\DeclareMathOperator\std{std}

\newcommand\erklaerungFehlerNotation{%
    In dieser Notation bedeutet \num{1.234 +- 0.005}, dass der Wert
    $\num{1.234} \pm \num{0.005}$ ist. Die Ziffern in Klammern sind die
    Fehlerangabe. Um den Fehler zu erhalten, wird diese von rechts über die
    Zahl gelegt, alle anderen Stellen werden auf 0 gesetzt.
}

\ihead{physik412 – Versuch \versuchsnummer}
\ifoot{M. Ueding, L. Lemmer}

\subject{Praktikumsprotokoll}
\title{Elektronische Übergänge in Atomen}
\subtitle{physik412 – Versuch \versuchsnummer}
\author{
Martin Ueding \\
\small{\href{mailto:mu@martin-ueding.de}{mu@martin-ueding.de}}
\and
Lino Lemmer \\
\small{\href{mailto:lino.lemmer@uni-bonn.de}{s6lilemm@uni-bonn.de}}
}
\publishers{Tutor: Jens Peter}

\setcounter{secnumdepth}{4}
\setcounter{tocdepth}{3}

\begin{document}

\maketitle

\begin{abstract}
    Im ersten Teil dieses Versuchs betrachten wir mit einem Fabry-Pérot-Etalon
    die, durch den Zeeman-Effekt entstehende, Aufspaltung einer Spektrallinie
    von Cadmium. Aus dieser bestimmen wir das Bohrsche Magneton.

    Im zweiten Teil führen wir den Frank-Hertz-Versuch durch und ermitteln aus
    diesem die Energie des ${}^1\text{D}_2\to{}^1\text{P}_1$-Übergangs von
    Quecksilber.
\end{abstract}

\tableofcontents

\chapter{Theorie}

\section{Energieniveaus von Elektronen im Atom}

\subsection{Ohne äußeres magnetisches Feld}

\subsection{Mit äußerem magnetischem Feld}

\subsection{Niveau-Übergänge}

\section{Natürliche Linienbreite und Linienverbreiterung}

\section{Fabry-Pérot-Etalon}

\section{Polarisationsfilter und $\frac{\lambda}4$-Plättchen}

\section{Hall-Sonde}

\section{Frank-Hertz-Versuch}

\subsection{Aufbau und Funktionsweise}

\subsection{Termschema von Quecksilber}

\chapter{Aufbau}

\chapter{Zeeman-Effekt}

\section{Durchführung}

Der Strom, bei der in transversaler Konfiguration die Linien zu unterscheiden
sind, ist \SI{1.9}{\ampere}. In longitudinaler Konfiguration ist der Strom
\SI{1.0}{\ampere}. Die longitudinalen Linien sind deutlich dunkler.

\section{Auswertung}

\begin{figure}[htbp]
    \centering
    \includegraphics[width=\linewidth]{Alles.pdf}
    \caption{%
    }
    \label{fig:}
\end{figure}

\begin{figure}[htbp]
    \centering
    \includegraphics[width=\linewidth]{Zoom.pdf}
    \caption{%
    }
    \label{fig:}
\end{figure}

\section{Diskussion}

\chapter{Frank-Hertz-Versuch}

\section{Durchführung}

\section{Auswertung}

\section{Diskussion}

\printbibliography

\end{document}

% vim: et spell spelllang=de tw=79
