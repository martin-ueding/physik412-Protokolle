% Copyright © 2013 Martin Ueding <dev@martin-ueding.de>

\input{../../header.tex}

\usepackage{csquotes}
\usepackage[section]{placeins}
\usepackage{booktabs}
\usepackage{pdflscape}
\usepackage{tikz}
\usetikzlibrary{arrows}

\newcommand\versuchsnummer{401}
\DeclareMathOperator\std{std}
\newcommand\skp[2]{\left\langle#1,#2\right\rangle}

\newcommand\erklaerungFehlerNotation{%
    In dieser Notation bedeutet \num{1.234 +- 0.005}, dass der Wert
    $\num{1.234} \pm \num{0.005}$ ist. Die Ziffern in Klammern sind die
    Fehlerangabe. Um den Fehler zu erhalten, wird diese von rechts über die
    Zahl gelegt, alle anderen Stellen werden auf 0 gesetzt.
}

\ihead{physik412 – Versuch \versuchsnummer}
\ifoot{M. Ueding, L. Lemmer}

\subject{Praktikumsprotokoll}
\title{Elektronische Übergänge in Atomen}
\subtitle{physik412 – Versuch \versuchsnummer}
\author{
Martin Ueding \\
\small{\href{mailto:mu@martin-ueding.de}{mu@martin-ueding.de}}
\and
Lino Lemmer \\
\small{\href{mailto:lino.lemmer@uni-bonn.de}{s6lilemm@uni-bonn.de}}
}
\publishers{Tutor: Jens Peter}

\setcounter{secnumdepth}{4}
\setcounter{tocdepth}{3}

\begin{document}

\maketitle

\begin{abstract}
    Im ersten Teil dieses Versuchs betrachten wir mit einem Fabry-Pérot-Etalon
    die, durch den Zeeman-Effekt entstehende, Aufspaltung einer Spektrallinie
    von Cadmium. Aus dieser bestimmen wir das Bohrsche Magneton.

    Im zweiten Teil führen wir den Frank-Hertz-Versuch durch und ermitteln aus
    diesem die Energie des ${}^1\text{D}_2\to{}^1\text{P}_1$-Übergangs von
    Quecksilber.
\end{abstract}

\tableofcontents

\chapter{Theorie}

\section{Energieniveaus von Elektronen im Atom}

\subsection{Ohne äußeres magnetisches Feld}

Löst man die Schrödingergleichung für ein Elektron im radialsymmetrischen
Potenzial des Atomkerns, erhält man aus dieser diskrete Werte für die Energie,
die von der Hauptquantenzahl $n\in\mathbb{N}$ und der Drehimpulsquantenzahl
$l\in\mathbb{N}^0$ abhängen. Für jedes $n$ gibt es $n-1$ mögliche Werte für
$l$. Ist eine $z$-Richtung vorgegeben, so ist die entsprechende Komponente des
Drehimpulses ebenfalls gequantelt, sie beträgt $L_z = \hbar m_l$, wobei $m_l
= -l,\dots,+l$.

Ist kein äußeres Feld vorgegeben, kann der Drehimpuls in beliebige Richtungen
zeigen, ohne dass sich die Energie des Elektrons ändert. Jeder Energiewert ist
also $2l+1$-fach entartet.

\subsection{Mit äußerem magnetischem Feld}

Durch den Drehimpuls erzeugt das Elektron ein magnetisches Moment
\begin{align*}
    \vec\mu &= -\frac{e}{2m_\text{e}}\vec L.
    \intertext{%
        In einem äußeren magnetischen Feld $\vec B=B \vec e_z$ ändert sich
        entsprechend die Energie
    }
    E_\text{mag} &= \skp{\vec\mu}{\vec B} \\
             &= -\frac{e}{2m_\text{e}}\skp{\vec L}{\vec B} \\
             &= -\frac{e}{2m_\text{e}}L_zB \\
             &= -\mu_\text{B}m_lB,
\end{align*}
wobei $\mu_\text{B}$ Bohrsches Magneton heißt. Die Aufspaltung der
Energie, also die Aufhebung der $m_l$-Entartung, heißt Zeeman-Effekt.
Kann der Gesamtspin vernachlässigt werden, spricht man vom normalen
Zeeman-Effekt, ist dies nicht so, muss eine Spin-Bahn-Kopplung beachtet
werden. Die Rechnung dazu funktioniert analog zur obigen Betrachtung,
führt nur zu einer anderen Energieaufspaltung.

\subsection{Niveau-Übergänge}
\label{ssec:Niveau}

Elektronen können unter Absorption bzw. Emission von Photonen das Energieniveau
wechseln. Dabei müssen bestimmte Auswahlregeln beachtet werden. Für uns wichtig
sind die Regeln $\Delta l = 1$ und $\Delta m = 0,\pm 1$. Die entsprechenden
Übergänge zwischen $l = 2$ und $l=1$ sind in Abbildung~\ref{fig:Übergänge}
dargestellt. Die Energie der Photonen, die bei gleichgefärbten Übergängen frei
werden, sind gleich, sodass hier eine Aufspaltung in drei Linien stattfindet.

\begin{figure}
    \centering
    \begin{tikzpicture}
        \draw[->, thick] (9,-2) to (-3,-2) to (-3,5) node[label=150:$E$] {};

        \draw[thick] (0,0) node[label=180:{$m=-1$}] {} -- (2,0) coordinate (l1m-1);
        \draw[thick] (0,.5) node[label=180:{$m=0$}] {} -- (2,.5) coordinate (l1m0);
        \draw[thick] (0,1) node[label=180:{$m=1$}] {} -- (2,1) coordinate (l1m1);
        \node[below] (l1) at (1,-2) {$l=1$};

        \draw[thick] (5,1.5) coordinate (l2m-2) -- (7,1.5)node[label=0:{$m=-2$}] {} ;
        \draw[thick] (5,2) coordinate (l2m-1) -- (7,2)node[label=0:{$m=-1$}] {} ;
        \draw[thick] (5,2.5) coordinate (l2m0) -- (7,2.5)node[label=0:{$m=0$}] {} ;
        \draw[thick] (5,3) coordinate (l2m1) -- (7,3)node[label=0:{$m=1$}] {} ;
        \draw[thick] (5,3.5) coordinate (l2m2) -- (7,3.5)node[label=0:{$m=2$}] {} ;
        \node[below] (l2) at (6,-2) {$l=2$};
        \draw[red, ->, thick] (l1m-1) to (l2m0);
        \draw[red, ->, thick] (l1m0) to (l2m1);
        \draw[red, ->, thick] (l1m1) to (l2m2);
        \draw[blue, ->, thick] (l1m-1) to (l2m-1);
        \draw[blue, ->, thick] (l1m0) to (l2m0);
        \draw[blue, ->, thick] (l1m1) to (l2m1);
        \draw[green, ->, thick] (l1m-1) to (l2m-2);
        \draw[green, ->, thick] (l1m0) to (l2m-1);
        \draw[green, ->, thick] (l1m1) to (l2m0);
    \end{tikzpicture}

    \caption{%
        Mögliche Übergänge zwischen $l=2$ und $l=1$.
    }
    \label{fig:Übergänge}
\end{figure}


\section{Natürliche Linienbreite und Linienverbreiterung}

In Abschnitt \ref{ssec:Niveau} ist die Rede von Linien, die durch diskrete
Energieniveaus hervorgerufen werden. Demnach müsste eine Linie nur genau aus
einer Wellenlänge bestehen. Dies kann jedoch in der Realität nicht beobachtet
werden, neben der Natürlichen Linienbreite, die durch eine endliche
Anregungsdauer hervorgerufen wird findet eine Linienverbreiterung statt.

Hat ein strahlendes Material eine Temperatur $T>\SI{0}{\kelvin}$, bewegen sich
die Teilchen ungeordnet in alle Richtungen. Je nach dem wie groß die Komponente
der Geschwindigkeit in Beobachtungsrichtung ist, findet eine entsprechend
starke Dopplerverschiebung der Wellenlänge statt. Die eigentlich scharfe Linie
wird so zu einer Verteilung von nah aneinander liegenden Wellenlängen. Sie
verbreitert sich also.

\section{Fabry-Pérot-Etalon}

Das Fabry-Pérot-Etalon besteht aus einen planparallelen lichtdurchlässigen
Material mit Brechungsindex $n>1$, dessen Oberflächen zum Beispiel durch
aufdampfen von Silber eine hohe Reflexivität haben.

Trifft ein monochromatischer Lichtstrahl mit Wellenlänge $\lambda$ unter dem
Winkel $\alpha$ auf das Etalon, wird er gebrochen und in diesem sehr häufig
reflektiert (siehe Abbildung~\ref{fig:Etalon}). Bei jeder Reflektion gelangt
ein kleiner Teil der Intensität aus dem Etalon und verlässt diesen wieder unter
dem Winkel $\alpha$. Ist der Gangunterschied zwischen zwei Teilstrahlen nun ein
vielfaches der Wellenlänge findet konstruktive Interferenz statt, gilt dies
nicht, heben sich die Teilstrahlen gegenseitig auf. Die Bedingung für
konstruktive Interferenz lässt sich aus Abbildung~\ref{fig:Etalon} geometrisch
herleiten. Sie lautet
\begin{align*}
    k\lambda &= 2d\cos\del{\beta} - 2d\tan\del{\beta}\cos\del{1-\alpha}.
    \intertext{%
        Über den Brechungsindex erhält man einen Ausdruck für den Winkel
        $\beta$:
    }
    \beta &= \arcsin\del{\frac{\sin{\alpha}}n}
    \intertext{%
        Dadurch ergibt sich für die Interferenzbedingung:
    }
    k\lambda &=
\end{align*}
\begin{figure}
    \centering
    \begin{tikzpicture}
        % Etalon
        \draw[thick] (-1,-4.5) rectangle (1,3.5);
        % Dicke
        \draw[thick, |<->|] (-1,-4.7) to node[below] {$d$} (1,-4.7);
        % Brechungsindizes
        \node[label=180:{$n>1$}, label=0:{$n=1$}] at (1,3.2) {};
        % Lot
        \draw[dashed]
        (-2.5,2.) -- (0.5,2.)
        (-0.5,-3.) -- (2.5,-3.);
        % Strahlengang
        \draw
        (-2.5,3.5) -- (-1,2.)
        -- (1,1) coordinate (a)
        -- (2.5,-.5)
        (a) -- (-1,0)
        -- (1,-1) coordinate (b)
        -- (2.5,-2.5)
        (b) -- (-1,-2)
        -- (1,-3)
        -- (2.5,-4.5);
        % Winkel
        \draw
        (-2.0,2.) arc (180:135:1.0cm) node[label=270:$\alpha$]{}
        (0.5,2.) arc (0:-27:1.5cm) node[label=115:{$\beta$}]{}
        (-0.5,-3.) arc (180:153:1.5cm) node[label=290:$\beta$]{}
        (2,-3.) arc (0:-45:1.0cm) node[label=90:{$\alpha$}]{};
    \end{tikzpicture}
    \caption{%
        Strahlengang im Fabry-Pérot-Etalon
    }
    \label{fig:Etalon}
\end{figure}

\section{Polarisationsfilter und $\frac{\lambda}4$-Plättchen}

Ein Polarisationsfilter oder Linear-Polarisator ist ein optisches Bauteil, mit
dem Licht mit mehreren Polarisationsrichtungen gefiltert werden kann.
Anschaulich wird dabei eine einfallende Welle in zwei Komponenten zerlegt, eine
parallel und eine senkrecht zur Polarisationsrichtung des Filters polarisierte.
Durchgelassen wird nur die parallele Komponente, das Licht ist nun linear
polarisiert.

Ein $\frac\lambda4$-Plättchen ist ein optisches Element, mit dem eine zirkulär
polarisierte Welle in eine linear polarisierte umgewandelt werden kann und
umgekehrt. Es besteht aus einem Material, welches in verschiedene Richtungen
unterschiedliche Brechungsindizes hat. Rechnerisch wird die Welle in zwei
orthogonale Komponenten aufgespalten, welche das Material unterschiedlich
schnell durchdringen. Dabei kommt es zu einer Verschiebung, die, bei richtiger
Wahl der Dicke, genau ein viertel der Wellenlänge entspricht.

\section{Hall-Sonde}

Durch den Hall-Effekt wird in einem stromdurchflossenen Leiterstück, welches
von einem Magnetfeld durchsetzt wird, eine Spannung, die Hall-Spannung erzeugt.
Diese Spannung ist proportional zur Senkrecht auf dem Strom stehenden
Komponente des Magnetfeldes:
\[
    B = \frac{U}{R_\text{HS}I}
\]
Die Hall-Konstante hängt dabei vom Material und von der Geometrie des
Leiterstücks ab. Ist sie bekannt, kann bei eingestelltem Strom die Spannung
vermessen und so das Magnetfeld bestimmt werden.

\section{Frank-Hertz-Versuch}

\subsection{Aufbau und Funktionsweise}
\label{ssec:Aufbau_FH}

\begin{figure}
    \centering
    \begin{tikzpicture}[thick]
        % Röhre
        \draw
            (-3,4) -- (-3,-4)
            arc(180:360:3)
            -- (3,4)
            arc(0:180:3);
        % Gegenkathode
        \draw (-3,4) node[left] {Gegenkathode} -- (4,4);
        % Anode
        \draw[dashed](-3,2) node[left] {Anode}-- (3,2);
        \draw (3,2) -- (4,2);
        % Glühkathode
        \draw (-3.5,-4) -- (-2.6,-4) -- (-2.5, -3.8);
        \foreach \x in {-2.5, -2.3, ..., 2.3}
            \draw (\x,-3.8) -- (\x+0.2,-4.2);
        \draw (2.5,-4.2) -- (2.6,-4) -- (4,-4);
        % Verschaltung
        \draw
            (-3.5,-4) -- (-3.5, -8) -- (-0.28, -8)
            (-.2,-8) circle(0.08) node[above] {\small{$+$}}
            (.2,-8) circle(0.08) node[above] {\small{$-$}}
            (0.28, -8) -- (4, -8) -- (4, -4) -- (4, -1.28)
            (4, -1.2) circle(0.08) node[left] {\small{$-$}}
            (4, -0.8) circle(0.08) node[left] {\small{$+$}}
            (4, -0.72) -- (4, 2.72)
            (4, 2.8) circle(0.08) node[left] {\small{$+$}}
            (4, 3.2) circle(0.08) node[left] {\small{$-$}}
            (4, 3.28) -- (4,4)
        ;
        \draw[fill]
            (4,-4) circle(0.05)
            (4,2) circle(0.05);
        % weitere Beschriftungen
        \node[below] at (0,-8) {$U_\text{Heiz}$};
        \node[right] at (4,-1) {$U_\text{Beschl}$};
        \node[right] at (4, 3) {$U_\text{Gegen}$};
        \node[below] at (0,-4.2) {Glühkathode};
        % Füllung
        \draw
        (-4,0) node[left] {Hg-Dampf} -- (-1,-1);
    \end{tikzpicture}
    \caption{%
        Aufbau des Frank-Hertz-Versuchs.
    }
    \label{fig:Aufbau_FH}
\end{figure}

\subsection{Termschema von Quecksilber}

\chapter{Aufbau}

\chapter{Zeeman-Effekt}

\section{Durchführung}

Der Strom, bei der in transversaler Konfiguration die Linien zu unterscheiden
sind, ist \SI{1.9}{\ampere}. In longitudinaler Konfiguration ist der Strom
\SI{1.0}{\ampere}. Die longitudinalen Linien sind deutlich dunkler.

\section{Auswertung}

\begin{figure}[htbp]
    \centering
    \includegraphics[width=\linewidth]{Alles.pdf}
    \caption{%
    }
    \label{fig:}
\end{figure}

\begin{figure}[htbp]
    \centering
    \includegraphics[width=\linewidth]{Zoom.pdf}
    \caption{%
    }
    \label{fig:}
\end{figure}

%< for I in I_list: >%
\begin{figure}[htbp]
    \centering
    %\includegraphics[width=.9\linewidth]{Fit-<< I >>.pdf}
    \caption{%
        Fit für \SI{<< I >>}{\ampere}.
    }
    \label{fig:<< I >>}
\end{figure}
%< endfor >%

\section{Diskussion}

\chapter{Frank-Hertz-Versuch}

\section{Durchführung}

\section{Auswertung}

\section{Diskussion}

\printbibliography

\end{document}

% vim: et spell spelllang=de tw=79
